\chapter{\textit{Continuation Passing Style} és kapcsolódó eszközök}

Az előző fejezet bevezette a generátorokat, melyeknek használatát lehetővé szeretnénk tenni \textit{Java}ban. Egyelőre azonban csak azt tudjuk, hogy \textit{mit} szeretnénk elérni, a \textit{hogyan} rejtve maradt. Felfedni ezt is csak lépésenként érdemes, továbbra is fokozatosan bevezetve a fogalmakat, melyekre a módszer támaszkodik. E fejezet a \textit{Continuation Passing Style} (továbbiakban \textit{CPS}), és néhány kapcsolódó eszköz ismertetését tartalmazza. Ezek birtokában már kifejezhetjük a dolgozat eredményét jelentő transzformációt és annak implementációját.

\section{\textit{Direct Style}}

A \textit{CPS} kifejtése előtt röviden vizsgáljuk meg ellentétét, a \textit{direct style}-t \cite{Danvy1994}, mely nem szorul hosszú bemutatásra, hiszen ez a programok írásának általánosan alkalmazott eszköze. Az ebben a stílusban írt függvények pusztán a formális paramétereiknek megfelelő argumentumokkal kerülnek meghívásra, futásuk befejeztét követően pedig visszatérnek a hívóhoz valamilyen értékkel (vagy bármilyen érték nélkül).

Friedman és Wand \citeyear{EssProgLan} gondolatmenetét követve, vegyük szemügyre a faktoriális kiszámítását végző különböző függvények viselkedését.

\begin{lstlisting}[language=JavaScript, caption={\textit{Direct style} faktoriálist meghatározó függvény}, captionpos=b, label=JSDSFactRec]
function factDSRec(n) {
    if (n === 0) {
        return 1;
    } else {
        return n * factDSRec(n - 1);
    }
}
\end{lstlisting}

A \ref{JSDSFactRec} kódrészlet egy rekurzív, \textit{direct style} függvény, mely \texttt{n} faktoriálisát határozza meg. A függvény működésének illusztrálásához vezessük le, hogyan kerül kiszámításra $4!$ értéke.

\begin{lstlisting}[language=JavaScript, caption={$4!$ kiszámítása \texttt{factDSRec} segítségével}, captionpos=b, numbers=none]
   factDSRec(4)
=  4 * factDSRec(3)
=  4 * (3 * factDSRec(2))
=  4 * (3 * (2 * factDSRec(1)))
=  4 * (3 * (2 * 1))
=  4 * (3 * 2)
=  4 * 6
=  24
\end{lstlisting}

Amíg a futás el nem éri a rekurzió alapesetét, új \texttt{factDSRec} hívások történnek, mindig eggyel kisebb \texttt{n} értékkel. Figyelemre méltó, hogy minden hívással együtt megjelenik egy szorzás is, mely csak a függvény visszatérését követően lesz elvégezhető.

\begin{lstlisting}[language=JavaScript, caption={\textit{Direct style} faktoriálist meghatározó függvény akkumulátorral}, captionpos=b, label=JSDSFactIter]
function factDSIter(n) {
    function factIterAcc(i, acc) {
        if (i === 0) {
            return acc;
        } else {
            return factIterAcc(i - 1, acc * i);
        }
    }

    return factIterAcc(n, 1);
}
\end{lstlisting}

A \ref{JSDSFactIter} kódrészlet két ponton különbözik jelentősen a \texttt{factDSRec} függvénytől. Az eltérések elemzését megelőzően azonban számoljuk ki ismét $4!$ értékét!

\begin{lstlisting}[language=JavaScript, caption={$4!$ kiszámítása \texttt{factDSIter} segítségével}, captionpos=b, numbers=none]
   factDSIter(4)
=  factIterAcc(4, 1)
=  factIterAcc(3, 4)
=  factIterAcc(2, 12)
=  factIterAcc(1, 24)
=  24
\end{lstlisting}

Az első, ami szembetűnik, hogy a számítás nem rendelkezik a \texttt{factDSRec} ``\textit{piramis}'' mintájával, mely a sorozatos hívások által kijelölt szorzásoknak volt köszönhető. A rekurzív hívások ebben az esetben is jelen vannak, azonban az előző függvény \textit{hátralevő} szorzásait az \texttt{acc} paraméterbe helyeztük. Ennek következtében nem szükséges a kijelölt műveletek folyamatos feljegyzése. Ezt a működést, a rekurzió ellenére \textit{iterative control behaviour}nak nevezzük \shortcite{EssProgLan}.

\section{\textit{Continuation Passing Style}}

A \textit{CPS}-t azt elmúlt században egymástól függetlenül többen is felfedezték, és különböző formában definiálták \cite{Reynolds1993}. Egy elegáns meghatározás a következő, mely pusztán két szabályt fektet le. A függvények

\begin{enumerate}
    \item sohasem térhetnek vissza, és
    \item formális paraméterlistájuk kiegészül egy \textit{continuation}nek nevezett paraméterrel \cite{MightCPS}.
\end{enumerate}

A stílus minden előnyös tulajdonsága a fenti szabályok alkalmazásának köszönhető. Viszont felmerülhet a kérdés, hogy ha egy függvény sohasem térhet vissza, akkor mégis hogyan képes értéket visszaadni? Sussman és Steele \citeyear{Sussman1975} erre a következő választ adja: ``\textit{\textellipsis in this continuation-passing programming style, a function always "returns" its result by "sending" it to another function}''. 

A függvény, aminek a visszatérési érték át lesz adva, nem más, mint a \textit{continuation}. Ez fogja jelképezni a még hátralevő számításokat, azaz hogy \textit{mi a teendő ezután} \cite{CompCont}.

\subsection{Példák \textit{CPS} függvényekre}

Amilyen egyszerű módon vezettük be a stílust, legalább annyira nehéz elképzelni, hogyan is működik. Ezért nézzünk két példát a \textit{CPS} alkalmazására!

\begin{lstlisting}[language=JavaScript, caption={\textit{CPS} összeadást végző függvény és meghívása}, captionpos=b, label=JSCPSAdd]
function addCPS(a, b, cont) {
    cont(a + b);
}

addCPS(10, 20, function print(n) {
    console.log(n);
});
\end{lstlisting}

A \ref{JSCPSAdd} kódrészlet először egy \texttt{addCPS} nevű függvényt definiál, melynek \textit{direct style} esetén csak két paramétere lenne, az összeadandó értékek. A második szabály értelmében azonban hozzá kellett adni egy harmadik paramétert is. Ez lesz a \textit{continuation}, aminek át kell adnunk az összeadás eredményét, hiszen az első szabály megtiltja, hogy a függvény visszatérjen vele. Ezután a függvény meghívásának módja már egyértelmű. Átadunk egy függvényt, mely paraméterül várja a kiszámított értéket, és tesz vele valamit, jelen esetben kiírja a konzolra.

Következzen egy összetettebb példa, a \texttt{factDSRec} függvény \textit{CPS}-transzformációja!

\begin{lstlisting}[language=JavaScript, caption={\textit{CPS} faktoriálist kiszámító függvény}, captionpos=b, label=JSCPSFact]
function factCPS(n, cont) {
    if (n === 0) {
        cont(1);
    } else {
        factCPS(n - 1, val => cont(n * val));
    }
}
\end{lstlisting}

Mire érdemes felfigyelni a függvény törzsében? Természetesen a \texttt{return} utasítások eltűntek, helyükre viszont annál érdekesebb kifejezések kerültek. A \texttt{return 1;} utasítást felváltotta a konstans érték átadása a \textit{continuation}nek, hasonlóan az \texttt{addCPS} függvényhez. Az eredeti \texttt{factDSRec} függvény ötödik sorának helyén álló hívás viszont már nem ilyen magától értetődő. Követhetőbbé válik a transzformáció folyamata, ha beiktatunk egy közbülső lépést.

\lstset{language=JavaScript, numbers=none, breaklines=true}

\begin{table}[h]
\captionsetup{justification=centering}
\centering
\begin{tabular}{p{0.33\linewidth} | p{0.33\linewidth} | p{0.33\linewidth}}
\begin{lstlisting}
return
    n * factDSRec(n - 1);
\end{lstlisting}&
\begin{lstlisting}
const val =
    factDSRec(n - 1);

return n * val;
\end{lstlisting}&
\begin{lstlisting}
factCPS(n - 1, 
 val => cont(n * val));
\end{lstlisting} 
\end{tabular}
\caption{A \texttt{return} utasítás transzformálásának lépései a \texttt{factDSRec} függvényben}  
\label{table:steps}
\end{table}

% reset lstset
\lstset{xleftmargin=15pt,
        basicstyle=\scriptsize,
        numbers=left,
        numbersep=5pt,
        numberstyle=\tiny\color{codegray},
        escapechar=@,
        aboveskip=2em,
        belowskip=2em,
        belowcaptionskip=2em}

A középső lépésben egy nevet rendelünk az eredetileg rekurzív függvényhívás eredményéhez, ezzel explicitté téve azt, hogy a szorzás elvégzése előtt egy függvényhívást kell végrehajtani. Pontosan ez jelenik meg a jobboldalon is. Meghívjuk a \texttt{factCPS} függvényt, az eredeti \textit{continuation} helyett egy új függvénnyel, melyben elvégezzük a szorzást. Utána a szorzás eredménye már továbbadható a kapott \textit{continuation}nek.

Az \texttt{addCPS} és a \texttt{factDSRec} függvény transzformációjának lépései valójában néhány szabályban összefoglalhatóak és általánosíthatóak. Egy ilyen transzformációs ``\textit{receptet}'' ad meg Friedman \citeyear{EssProgLan}.

\subsection{Programvezérlés \textit{CPS} segítségével}

Az előző példák kevésbé szóltak a \textit{CPS} mellett, hiszen egy viszonylag egyszerű faktoriális függvény átírása is egy elsőre ránézésre nem teljesen egyértelmű kódrészletet eredményezett. Ha viszont eltekintünk a \textit{continuation} kezelése okozta komplexitástól, akkor megcsillan a \textit{CPS} egyik legjobb tulajdonsága. A programvezérlés útja explicitté vált, a műveletek végrehajtásának sorrendjével együtt \shortcite{CompCont, EssProgLan}. A \ref{table:steps} táblázat lépései világosan mutatják, ahogy a szorzás elvégzéséhez szükséges műveletek és azok sorrendje explicit kifejezésre került. Ugyanígy, a \textit{continuation}ök vizsgálatával pontosan megmondhatjuk, sőt kontrollálhatjuk, hogy a vezérlés merre fog tovább haladni.

Ez a jellemző olyan erőt ad a programozók és a fordítóprogramok készítőinek kezébe, mely primitívebb vezérlési szerkezetek mellett akár korutinok megvalósítását is lehetővé teszi \shortcite{Haynes1984}. Mi több, valójában tetszőleges vezérlési elv implementálására alkalmas apparátust kínál a \textit{CPS} \cite{Lippert2010}. 

Ennek ismeretében tegyük fel, hogy a \textit{JavaScript} nyelv egy olyan változatát kell használnunk, mely nem rendelkezik a kivételek kezelésére alkalmas \texttt{try-catch} struktúrával. Lippert \citeyear{Lippert2010} \textit{C\#}-ban készített transzformációját alapul véve, szeretném megmutatni, hogy a \textit{CPS} egy olyan eszköz, ami lehetővé teszi, hogy elméleti \textit{JavaScript} nyelvünk hiányosságát pótoljuk.

Az első lépés a meglevő \texttt{cont} paraméter mellé egy második, \texttt{err} nevű \textit{continuation} paraméter hozzáadása. Ezt a változtatást a kivételkezelés indokolja. A program futása során a vezérlés rendes irányát megszakíthatja egy kivétel kiváltása. Ilyen esetben a futásnak egy kivételkezelő blokkban kell folytatódnia. Az újonnan bevezetett \textit{continuation} fogja szimbolizálni a kivételes helyzetekben választandó futási irányt.

Kivétel a legtöbb programozási nyelvben egy \texttt{throw} nevű utasítással váltható ki. Ezzel összehangban a modellben is ezt a nevet fogjuk használni.

\begin{lstlisting}[language=JavaScript, caption={A \textit{throw} utasítás \textit{CPS}-ben}, captionpos=b, label=JSThrow]
function Throw(cont, err) {
    err();  
}
\end{lstlisting}

A \ref{JSThrow} kódrészlet \texttt{Throw} függvénye mindössze meghívja a kivételek kezelésére szolgáló \textit{continuation}t.

A teljes szerkezet modellezésére szolgáló függvény elkészítése előtt szükséges a vezérlés vizsgálata. Mikor merre kell folytatódnia a programvégrehajtásnak?

\begin{itemize}
    \item Ha a \texttt{try} blokkban nem történik \texttt{Throw} hívás, akkor a futásnak \texttt{try-catch} utáni utasítással kell folytatódnia, ami az eredetileg kapott \texttt{cont} paramétert jelenti.
    \item Amennyiben a \texttt{try} blokkon belül kivétel történik, akkor viszont a \texttt{catch} blokkot kell végrehajtani. Ezt kell felhasználni a \texttt{try} blokk \texttt{err} paramétereként.
    \item A \texttt{catch} blokknak is szüksége van két \textit{continuation}re. Az előző két pontban használt gondolatmenetet alkalmazhatjuk itt is. Ha nem történik kivétel, akkor az eredeti \texttt{cont} paraméter jelentette utat követjük. Kivétel bekövetkeztekor pedig a tartalmazó (eggyel magasabb szintű) \texttt{catch} blokkra támaszkodunk, amit a kapott \texttt{err} paraméter segítségével érünk el.
\end{itemize}

Összegezve a fentieket, az implementáció a következő:

\begin{lstlisting}[language=JavaScript, caption={A \textit{try-catch} blokk \textit{CPS}-ben}, captionpos=b, label=JSTryCatch]
function TryCatch(tryBlock, catchBlock, cont, err) {
    tryBlock(cont, () => catchBlock(cont, err));
}
\end{lstlisting}

Nézzük meg a \texttt{TryCatch} működését a jól ismert \texttt{factCPS} egy módosított változatában!

\begin{lstlisting}[language=JavaScript, caption={Faktoriálist kiszámító függvény \textit{CPS}-ben, \texttt{try-catch} szerkezettel}, captionpos=b, label=JSCPSFactTryCatch]
function factCPSTryCatch(n, cont, err) {
    TryCatch(
        (cont, err) => {
            if (n < 0) {
                Throw(cont, err); 
            } else if (n === 0) {
                cont(1);
            } else {
                factCPS(n - 1, val => cont(n * val), err);
            }    
        }, 
        (cont, err) =>  cont(-1),
        cont, err
    );
}
\end{lstlisting}

A \ref{JSCPSFactTryCatch} kódrészlet egy \texttt{try-catch} szerkezettel egészíti ki a \ref{JSCPSFact} kódrészlet függvényét. Amennyiben $n$ negatív, a \texttt{Throw} egy kivételt vált ki. Ennek hatására a vezérlés a \texttt{TryCatch} függvény második paramétereként átadott kivételkezelő kóddal fog folytatódni. Ez pedig az ismert módon ``\textit{visszatér}'' mínusz eggyel.

\section{\textit{Tail Call Optimization}}

Eddig a \textit{CPS} programvezérlésre kifejtett hatása volt előtérben. Ha újra megvizsgáljuk a kódrészleteket, a stílus egy egészen más aspektusára lehetünk figyelmesek. A \ref{JSCPSFactTryCatch} kódrészletben a vezérlés minden lehetséges ágának utolsó utasítása egy függvényhívás. Ez a \textit{CPS} első szabályának következménye. Mivel a függvények sohasem térnek vissza, ezért a függvényhívás után álló utasítások nem fognak végrehajtásra kerülni. Hívás még egy másik függvény argumentuma sem lehet, az ilyen helyzeteket a \textit{continuation} megfelelő szervezésével kell megoldani. Tehát minden hívás úgynevezett \textit{tail call} \cite{CompCont}.

Bár a \textit{CPS}-transzformáció következménye az ilyen hívások megjelenése, a \textit{tail call} a \textit{CPS}-től teljesen független fogalom, egyszerűen olyan hívást jelent, mely egy függvény utolsó utasítása. Ha ez a hívás egy rekurzív hívás, akkor \textit{tail recursion}ről beszélünk (bár a szakirodalomban gyakran mindkét esetet így nevezik). \textit{Tail call} esetén a hívó rutinhoz nincs miért visszatérni, hiszen neki csak egyetlen dolga maradt: visszaadni a vezérlést az őt hívónak. Általánosan, ha egy $f$ rutin \textit{tail call} formájában hívja $g$-t, akkor $g$-nek már nem kell visszatérnie $f$-hez, elegendő $f$ hívójához visszatérnie \cite{CompCont}. 

Hagyományos esetben, amikor egy szubrutinhívás történik, meg kell jegyezni, hogy a hívás után hova kell visszatérni. Többek között erre szolgál a verem (\textit{stack}). Minden hívás egy újabb \textit{stack frame} létrehozását jelenti \cite{EssProgLan}. Egymást követő számtalan hívás után a verem túlcsordulhat, ez a \textit{stack overflow}. A \textit{tail call optimization} (néha \textit{tail recursion elimination}, a változatos elnevezések miatt) a verem túlcsordulásának elkerülésére szolgáló, az előző felismeréseken alapuló technika. Többféleképpen megvalósítható, a \textit{CHICKEN} \textit{Scheme} fordító például Baker 1994-es javaslatát alkalmazza \cite{ChickenCompilation}. Baker ajánlásában a verem a soha vissza nem térő hívások következtében folyamatosan növekszik, mielőtt azonban túlcsordulna, a felesleges \textit{stack frame}-ek egy \textit{garbage collector} segítségével törlésre kerülnek \cite{CheneyOnTheMTA}.

Összegezve, amennyiben egy függvény \textit{tail call}-lal ér véget, a \textit{direct style} kapcsán említett \textit{iterative control behaviour} automatikusan megvalósul. Ezt pedig kihasználhatjuk a \textit{tail call optimization} alkalmazásával, ami elhárítja a verem túlcsordulásának veszélyét.

\subsection{\textit{Trampolining}}

 A \textit{tail call optimization} nem csak olyan összetett technikákkal vihető véghez, mint a Baker-féle \textit{Cheney on the M.T.A.} Egy könnyedén megvalósítható megoldást kínál a \textit{trampolining}.

 Adott egy legkülső függvény, a \textit{trampoline}. Amikor egy függvény szeretne meghívni egy másikat, azt a \textit{trampoline} közbeiktatásával teheti meg. Ez azt jelenti, hogy a hívónak vissza kell adnia valamilyen, a meghívandó függvényre utaló adatot, mely alapján a \textit{trampoline} el tudja végezni a hívást \cite{Schinz2001}. A \textit{verem} mérete így nem fog minden határon túl nőni, hanem alternálva, egyszer nő, egyszer csökken. Erről a szemléletes fel-le mozgásról kapta a nevét az eljárás.

 A példa bemutat egy visszaszámláló függvényt, ami egyre kisebb értékkel hívja meg önmagát, míg el nem éri a nullát.

\begin{lstlisting}[language=JavaScript, caption={\textit{Trampolining} megvalósítása \textit{JavaScript}ben}, captionpos=b, label=JSTrampoline]
function trampoline(start) {
  let bounce = start;

  while (bounce && bounce.func) {
    bounce = bounce.func(...bounce.args);
  }
}

function counter(n) {
  if (n > 0) {
    console.log(n);

    return { 
      func: counter, 
      args: [n - 1]
    };
  } else {
    return null;
  }
}
\end{lstlisting}
 
 A \ref{JSTrampoline} kódrészlet kulcsszereplője a \texttt{trampoline} függvény ciklusa. Ez mindaddig fut, amíg a legutóbb végrehajtott függvény visszaad egy megfelelő objektumot. Ebben az objektumban egy, a következőnek meghívandó függvényre mutató referencia, valamint az átadandó argumentumok szerepelnek, egy tömb formájában.

 Ugyanezt a működést megvalósító, \textit{trampoline} nélküli kód, egy megfelelően nagy szám megadásával a teljes vermet el fogja használni. A példában szereplő \texttt{counter} viszont tetszőleges értékkel megbirkózik, hiszen a tárigénye korlátozott.