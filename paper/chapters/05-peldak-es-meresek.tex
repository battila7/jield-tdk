\chapter{Példák és mérések}\label{ch:peldakEsMeresek}

A fejezetben példák két típusán keresztül kívánom szemléltetni a generátorok kínálta sokszínű lehetőségeket. Elsőként a hagyományosabb felhasználási módot bemutató példák szerepelnek, melyek valamilyen sorozat előállítását valósítják meg. Ezek a példák nem csak a \textit{Jield} könyvtárral, hanem kézzel írt generátorokat is magukban foglalnak, lehetőséget biztosítva az összehasonlításra. Objektív szempont a kódok teljesítménye, melynek mérése a mikromérések (\textit{microbenchmark}) készítésére alkalmas \textit{JMH} könyvtárral történt. Emellett olyan, szubjektívebb szempontot tekintve is összevetésre kerültek a megvalósítások, mint az olvashatóság és áttekinthetőség.

A második csoport a generátorok vezérlésre kifejtett hatását kiaknázó kódrészleteket tartalmaz. Az összehasonlítást nélkülözve, kizárólag a kidolgozott implementáción alapuló példák szerepelnek, melyek azt hivatottak bemutatni, hogy elrugaszkodva a generátorok megszokott sorozatképző viselkedésétől, alkalmazásukkal bámulatos vezérlési út hozható létre.

\section{Konfiguráció}

Az összehasonlítást tartalmazó példák előtt tekintsük át a teljesítményteszthez használt konfigurációt, hiszen egy teljesítményteszt eredményének reprodukálásához elengedhetetlen a használt paraméterek pontos ismerete. Ez magában foglalja a mérésekhez alkalmazott keretrendszer, a \textit{JMH} beállításait és a futtató környezet szoftver- és hardverkonfigurációját is. Előbbit az \ref{table:jmhparams}, utóbbit az \ref{table:envparams} táblázat tartalmazza.

\begin{table}[h]
\captionsetup{justification=centering}
\centering
  \begin{tabular}{|| l | c ||}
  \hline
  Paraméter & Érték \\
  \hline \hline
  JMH Version                        & 1.18 \\
  \hline
  Warmup iterations                  & 10 \\
  Time per warmup iteration          & 1 sec \\
  \hline
  Measurement iterations             & 10 \\
  Time per measurement iteration     & 1 sec \\
  \hline
  Forks                              & 5 \\
  \hline                               
  Mode                               & Average time \\
  \hline
  \end{tabular}
\caption{A mérések során alkalmazott \textit{JMH} paraméterértékek}  
\label{table:jmhparams}
\end{table}

\begin{table}[h]
\captionsetup{justification=centering}
\centering
  \begin{tabular}{|| l | c ||}
  \hline
  Paraméter & Érték \\
  \hline \hline
  Processzor                  & Intel Core i5-6500, 3.2 GHz \\
  Memória                     & 16 GB, DDR4 \\
  \hline
  Operációs rendszer                        & Windows 10 64 bit \\
  \hline
  JRE verzió                  & 1.8.0u121-b13 \\
  JVM verzió                  & HotSpot 25.121-b13 \\
  \hline
  \end{tabular}
\caption{A méréseket futtató környezet paraméterei}  
\label{table:envparams}
\end{table}

Az \ref{table:jmhparams} táblázat a \textit{JMH} beállításait a keretrendszerben használt nevükkel sorolja fel. Összefoglalva, minden tényleges mérést 10, egyenként 1 másodperc hosszúságú bemelegítő iteráció előzött meg. A bemelegítést követően 10, szintén 1 másodpercig tartó ismétlés alatt a végrehajtáshoz szükséges átlagidő került mérésre. Ez az eljárás 5 független \textit{JVM}-en lett elvégezve.

\section{Fibonacci-sorozat}

Az első példát a Fibonacci-sorozat elemeit generáló metódusok alkotják. A sorozat elemeinek kiszámítását végző algoritmus azonos, a három kódrészlet közötti különbséget az algoritmust működtető mechanizmus jelenti. Az összehasonlított megvalósítások közül kettő szerepel az \ref{FibComparison} kódrészletben. A harmadik, az \texttt{intStreamGenerate} annyiban tér el a \texttt{streamGenerate} metódustól, hogy az \texttt{int} típusra specializált \texttt{IntStream} és \texttt{IntSupplier} osztályokat alkalmazza. Ez az eltérés bár aprónak tűnhet, azonban látni fogjuk, hogy a kódrészlet teljesítményére jelentős hatással van.

\begin{center}
\begin{mdframed}[topline=true]
\begin{minipage}[t]{0.45\textwidth}
\begin{lstlisting}[language=Java, breaklines=true, escapechar=!]
@Generator
public Stream<Integer> jield() {
  int a = 0, b = 1;

  while (true) {
    int temp = a;
    a = b;
    b = a + temp;

    return temp;
  }
}
\end{lstlisting}
\end{minipage} 
\begin{minipage}[t]{0.55\textwidth}
\begin{lstlisting}[language=Java, breaklines=true, escapechar=!]
public Stream<Integer> streamGenerate() {
  return Stream.generate(
    new Supplier<Integer>() {
      private int a = 0;
      private int b = 1;

      @Override
      public Integer get() {
        int temp = a;
        a = b;
        b = a + temp;
        return temp;
      }
    });
}
\end{lstlisting} 
\end{minipage}
\end{mdframed}
\captionof{lstlisting}{A Fibonacci-sorozat előállítása \textit{Jield} és \texttt{Stream.generate} segítségével}
\label{FibComparison}
\end{center}

\subsection{Olvashatóság}

Sorok számát tekintve nincs jelentős különbség az \ref{FibComparison} kódrészlet két oldala között. Azonban, míg a \texttt{jield} metódus szinte kizárólag egy elem előállítását adja meg, addig a \texttt{streamGenerate} szükségszerűen tartalmaz a generátor-mechanizmusra vonatkozó implementációs részleteket. Annak ellenére, hogy a két kód lényegét tekintve azonos, a \texttt{jield} könnyebben áttekinthető, hiszen csak egy végtelen ciklusból áll. A sorozat egy elemét kiszámoló kód egy végtelen ciklusban került elhelyezésre, lezárva egy \texttt{return} utasítással. Ez a recept vagy minta természetesen más sorozatok esetén is felhasználható.

Érdemes megemlíteni, hogy bár a \texttt{Stream.generate} metódus meghívható egy lambda függvénnyel is, jelen esetben mindenképpen névtelen osztályt kell létrehozni, a generátor állapotának megőrzése érdekében. Igazából emiatt szenved hátrányt a \texttt{streamGenerate} olvashatósága.

\subsection{Teljesítmény}

Az implementációk teljesítményét a Fibonacci-sorozat első 1, 10 és 100 elemének generálásával mértem, a fejezet elején említett beállítások mellett. A tesztek előtt természetesen rendelkeztem egy elvárással a lehetséges sorrendet illetően. Úgy gondoltam, hogy az \texttt{intStreamGenerate} biztosan meg fogja előzni a \texttt{streamGenerate} metódust, hiszen csak primitíveken dolgozik. A \texttt{jield} futási idejét azonban tapasztalatok hiányában nem tudtam megbecsülni.

\begin{table}[h]
\captionsetup{justification=centering}
\centering
  \begin{tabular}{|| c | c | c ||}
  \hline
  Metódus neve & Generált elemek száma & Átlagos futási idő ($\mu\mathrm{s}$) \\
  \hline \hline
  \multirow{3}{8.8em}{\texttt{jield}} & 1 & 0.136 \\
  & 10 & 0.584 \\
  & 100 & 5.079 \\
  \hline
  \multirow{3}{8.8em}{\texttt{streamGenerate}} & 1 & 0.039 \\
  & 10 & 0.116 \\
  & 100 & 0.923 \\
  \hline
  \multirow{3}{8.8em}{\texttt{intStreamGenerate}} & 1 & 0.037 \\
  & 10 & 0.096 \\
  & 100 & 0.679 \\
  \hline
  \end{tabular}
\caption{A Fibonacci-sorozatot generáló metódusok átlagos futási idejének összehasonlítása}  
\label{table:fibComp}
\end{table}

Az \ref{table:fibComp} táblázat adatai nem nevezhetőek meglepőnek. A generált elemek számával minden esetben együtt nő a végrehajtáshoz szükséges idő. A \texttt{jield} a leglassabb a három metódus közül, valószínűleg a \textit{trampolining} és a sok lambda függvény miatt. 100 elem generálásakor átlagosan megközelítőleg $7,5$-szer tovább fut, mint az \texttt{intStreamGenerate}, mely a várakozásoknak megfelelően megelőzte a \texttt{streamGenerate} metódust.

\section{\textit{Iterable} elemeinek ismétlése}

A második példa metódusai egy olyan sorozatot képeznek, mely egy \texttt{Iterable} elemeiből áll, és két paraméterrel befolyásolható. A \texttt{times} azt adja meg, hogy az \texttt{Iterable} elemeit hányszor állítsa elő a generátor, az \texttt{each} pedig azt, hogy az így kapott sorozat minden eleme hányszor legyen megismételve. Csak két generikus implementáció került összehasonlításra, az egyik a \textit{Jield}-féle generátor, a másik pedig a \texttt{Stream.generate} metódust veszi alapul.

\begin{lstlisting}[language=Java, caption={Elemek ismétlése \textit{Jield} segítségével}, escapechar=!, captionpos=b, belowskip=1em, belowcaptionskip=0em]
@Generator
public <T> Stream<T> jield(Iterable<T> iter, int times, int each) {
  for (int j = 0; j < times; ++j) {
    for (T element : iter) {
      for (int i = 0; i < each; ++i) {
        return element;
      }
    }
  }
}
\end{lstlisting}

\begin{lstlisting}[language=Java, caption={Elemek ismétlése \texttt{Stream.generate} használatával}, escapechar=!, captionpos=b, aboveskip=1em, label=repStreamGenerate]
public <T> Stream<T> streamGenerate(Iterable<T> iter, int times, int each) {
  return Stream.generate(
     () -> StreamSupport.stream(iter.spliterator(), false)
                        .map(t -> Collections.nCopies(each, t))
                        .flatMap(Collection::stream))
    .limit(times)
    .flatMap(Function.identity());
}
\end{lstlisting} 

\subsection{Olvashatóság}

A kódrészletek hosszában ismét nem tapasztalható jelentős eltérés. A két megvalósítás ennél jobban azonban nem is különbözhetne! A \texttt{jield} metódus pusztán ciklusokból építkezik, míg a \texttt{streamGenerate} a \texttt{Stream} és \texttt{StreamSupport} osztályok szolgáltatásait használja. Ugyan nem jelenthető ki egyértelműen, hogy az egyik megoldás olvashatóságot tekintve jobb lenne a másiknál, azonban az implementációk értelmezéséhez szükséges előismeret és háttértudás markánsan eltér. 

A \textit{Jield} a megszokott vezérlési szerkezeteket alkalmazva teszi lehetővé generátorok elkészítését. A programozónak csak annyit kell megjegyeznie, hogy a \texttt{return} többé nem csak visszatérést, hanem visszatérést és felfüggesztést jelent egyben. Hozzászokva ehhez a gondolathoz, már mindent tud, amit a \textit{Jield} megkövetel a generátorok írásához. A \textit{Stream API} használata, mint az \ref{repStreamGenerate} kódrészletből is látszik, nem ilyen egyszerű. Ez nem is meglepő, figyelembe véve, hogy a \textit{Stream API} nem kifejezetten generátorok elkészítését célozza, mint a \textit{Jield}.

\subsection{Teljesítmény}

A mérésekhez kiválasztottam három-három \texttt{times} és \texttt{each} paraméterértéket, és ezek összes lehetséges kombinációjára lefuttattam a megvalósításokat. Az átadott \texttt{Iterable} minden esetben egy három elemű, \texttt{String} objektumokból álló lista volt, az \texttt{Arrays.asList} metódussal összeállítva. Mivel mindkét implementáció azonos \texttt{Iterable}-t használ, így az feltehetően azonos mértékben befolyásolja a teljesítményüket is.

\begin{table}[h]
\captionsetup{justification=centering}
\centering
  \begin{tabular}{|| c | c | c | c ||}
  \hline
  Metódus neve & \texttt{times} & \texttt{each} & Átlagos futási idő ($\mu\mathrm{s}$) \\
  \hline \hline
  \multirow{9}{8.8em}{\texttt{jield}} & 1 & 1 & 0.542  \\
  & 10 & 1 & 4.653  \\
  & 100 & 1 & 44.855  \\
  & 1 & 10 & 2.164  \\
  & 10 & 10 & 20.481  \\
  & 100 & 10 & 203.088  \\
  & 1 & 100 & 18.727 \\
  & 10 & 100 & 183.869  \\
  & 100 & 100 & 1845.712 \\
  \hline
  \multirow{9}{8.8em}{\texttt{streamGenerate}} & 1 & 1 & 0.291  \\
  & 10 & 1 & 2.372  \\
  & 100 & 1 & 22.750  \\
  & 1 & 10 & 0.519  \\
  & 10 & 10 & 4.459  \\
  & 100 & 10 & 44.057  \\
  & 1 & 100 & 2.666  \\
  & 10 & 100 & 24.469  \\
  & 100 & 100 & 245.874  \\
  \hline
  \end{tabular}
\caption{Elemek ismétlését végző metódusok összehasonlítása}  
\label{table:repComp}
\end{table}

\pagebreak

Előzetesen azt vártam, hogy a \texttt{jield} lassabb lesz, mint a \texttt{streamGenerate}. Ez az elvárás be is igazolódott, amint az az \ref{table:repComp} táblázat adataiból kiolvasható. Legrosszabb esetben, a paraméterek maximális értéke esetén ezúttal is $7,5$-szeres különbség figyelhető meg, a \texttt{streamGenerate} javára. 

\section{Mikroszálak, szimmetrikus korutinok}

A hagyományos, valóban generátorszerű működést megvalósító kódrészletek után következzenek azok a példák, melyek a generátorok aszimmetrikus korutin voltára építenek. Két példát kívánok bemutatni, melyek azonban együtt szerepelnek, hiszen implementációjuk alapját ugyanaz a kód adja. 

A \textit{Jield} a verem túlcsordulásának elkerülésére a 3. fejezetben ismertetett \textit{trampolining} technikát használja az egyes generátorok esetén. Mi történik azonban ha a generátorok meghívása szintén egy \textit{trampoline}-ból (vagy ahhoz hasonló konstrukcióból) történik, és azok valamilyen vezérlőjelet adnak vissza? E vezérlőjeltől függően lehetségessé válik szimmetrikus korutinok és kooperatív multitaszking megvalósítása.

\subsection{Az ütemező}

A generátorok vezérlésének logikáját tartalmazó, eddig \textit{trampoline}-nak nevezett szerkezetre a továbbiakban ütemezőként fogok hivatkozni. Az ütemező a legutóbb meghívott generátor által visszaadott vezérlőjelnek megfelelően folytatja működését. A vezérlőjelek a következőek lehetnek:

\begin{itemize}
  \item \textit{CONTINUE}: Az ütemező válasszon ki egy másik generátort legközelebbi meghívásra, valamilyen kiválasztási algoritmus szerint.
  \item \textit{REMOVE}: Az ütemező távolítsa el a jelet visszaadó generátort az ütemezett generátorok közül.
  \item \textit{QUIT}: Az ütemező fejezze be a működését.
  \item \textit{continueWith(id)}: Az ütemező az \textit{id} azonosítójú generátort válassza ki következőnek futásra.
\end{itemize}

Pusztán a \textit{CONTINUE} vezérlőjel már elegendő kooperatív multitaszking megvalósításához. A szimmetrikus korutinok modellezését pedig a \textit{continueWith} jel teszi lehetővé, mely biztosítja a generátorok számára az ütemezést megkerülve a saját futásuk felfüggesztését, és egy másik generátor futásának folytatását.

Az ütemező legfontosabb metódusát az \ref{execute} kódrészlet tartalmazza, mely mindaddig fut, amíg van rendelkezésre álló generátor. A generátorok egy \texttt{SchedulableMicrothread} nevű osztályba vannak becsomagolva, ami a tényleges generátoron felül egy azonosítót és egy prioritást tartalmaz. Az \texttt{execute} metódus a legutolsónak végrehajtott generátor által visszaadott vezérlőjel és a \texttt{getNextThread} metódus segítségével választja ki a következőnek végrehajtandó generátort. Ezután, ezt meghívja, majd újra kiválaszt egy szálat, és így tovább.

\begin{lstlisting}[language=Java, caption={A generátorok ütemezését végző metódus}, escapechar=@, captionpos=b, aboveskip=1em, label=execute]
public void execute() {
    Signal previousSignal = CONTINUE;
    SchedulableMicrothread previousThread = null;

    while (!threadMap.isEmpty()) {
        Optional<SchedulableMicrothread> threadOptional = getNextThread(previousThread, previousSignal);

        if (threadOptional.isPresent()) {
            previousThread = threadOptional.get();

            if (previousThread.getThreadIterator().hasNext()) {
                previousSignal = previousThread.getThreadIterator().next();
            } else {
                threadMap.remove(previousThread.getIdentifier());

                previousSignal = CONTINUE;

                previousThread = null;
            }
        } else {
            return;
        }
    }
}
\end{lstlisting} 

Természetesen a bemutatott viselkedés csak egy lehetséges implementációja az ütemező ötletének, mely sok részletben megváltoztatható, továbbfejleszthető. Hozzáadhatók például további vezérlőjelek, vagy biztosítható a generátorok közötti adatcsere.

\subsection{Kooperatív multitaszking mikroszálakkal}

Ahogy az ütemező elnevezésből, valamint az \ref{execute} kódrészlet azonosítóiból is kitűnik, az \texttt{execute} metódus elsősorban úgynevezett mikroszálakra van felkészítve. Ezek a mikroszálak valójában generátorok, melyek maguk döntik el, hogy mikor adják vissza a vezérlést az ütemezőnek, így kooperatív (nem-preemptív) multitaszkingot megvalósítva. A vezérlés visszaadása a \texttt{return} utasítás segítségével történik, az említett vezérlőjelek valamelyikének kíséretében.

\begin{lstlisting}[language=Java, caption={Kooperatív mikroszálak}, escapechar=!, captionpos=b, aboveskip=1em, label=cooperative]
public static class EvenCounter implements Microthread {
    @Override @Generator
    public Stream<Signal> execute() {
        for (int i = 0; i < 10; i += 2) {
            System.out.println(i);
            return CONTINUE;
        }
    }
}
public static class OddCounter implements Microthread {
    @Override @Generator
    public Stream<Signal> execute() {
        for (int i = 1; i < 10; i += 2) {
            System.out.println(i);
            return CONTINUE;
        }
    }
}
\end{lstlisting} 

Az \ref{cooperative} kódrészlet két egyszerű mikroszálat mutat be, melyek a páros, valamint a páratlan számokat írják ki a konzolra. Azonos prioritással futtatva ezeket a mikroszálakat, a számok felváltva fognak a konzolra kerülni. Azonban a mikroszálat leíró osztály nem tartalmaz sem azonosítót, sem pedig prioritást. Ezeket csak akkor szükséges megadni, amikor az osztály egy példányát egy ütemező alá rendeljük. Ennek köszönhetően ugyanazon mikroszál több példánya is futhat egy ütemező felügyelete alatt, akár különböző prioritásokkal.

\subsection{Szimmetrikus korutinok}

A korutinok két típusa közti különbség röviden összegezhető annyiban, hogy egy aszimmetrikus korutinnak egy ponton mindenképpen vissza kell adnia a vezérlést az őt hívó kódnak, míg szimmetrikus korutinok esetén ez a korlátozás nem áll fenn. Érezhető, hogy ez a különbség nem olyan jelentős, s ez valóban így is van. A szakadék mindössze egy ütemezővel áthidalható.  Amikor a generátor (azaz aszimettrikus korutin) visszaadja a vezérlést, akkor kijelöli valamilyen módon, hogy mely másik generátorral folytatódjon a végrehajtás. Az előző ütemező implementációt alkalmazva ez a \textit{continueWith} vezérlőjel használatával történik.

\begin{lstlisting}[language=Java, caption={Számlálás szimmetrikus korutinokkal}, escapechar=!, captionpos=b, aboveskip=1em, label=symmetric]
public static class EvenCounter implements Microthread {
    @Override
    @Generator
    public Stream<Signal> execute() {
        for (int i = 0; i < 10; i += 2) {
            System.out.println(i);
            return continueWith("odd");
        }
    }
}

public static class OddCounter implements Microthread {
    @Override
    @Generator
    public Stream<Signal> execute() {
        for (int i = 1; i < 10; i += 2) {
            System.out.println(i);
            return continueWith("even");
        }
    }
}
\end{lstlisting} 

Az \ref{symmetric} kódrészlet két szimmetrikus korutint tartalmaz, melyek a megelőző kódrészlethez hasonlóan egyszerű számláló viselkedést valósítanak meg. Azonban nem hagyatkoznak az ütemezőre, hanem közvetlenül hívják egymást azonosítók segítségével. Bár a példa rendkívül egyszerű, kiválóan rávilágít a szimmetrikus korutinok lényegére, azaz hogy az egyes korutinok tetszőlegesen hívhatják egymást, mely hívások közt a legutolsó felfüggesztés helye és a lokális változók értéke megmarad.

\subsection{Összehasonlítás}

Érdemes az előző két példát összehasonlítani, hiszen bár a felszínen hasonlónak tűnhetnek, jobban már nem is különbözhetnének. Mivel az aszimmetrikus korutinok adják mindkét példa mögöttes mechanizmusát, ezért a lokális változók értékei, és a felfüggeszthető-újraindítható működés mindkét kódrészlet jellemzője. Az igazi eltérést az ütemező szerepe jelenti. Míg a mikroszálak esetén az ütemező aktív, azaz valóban döntéseket hoz a vezérlést illetően, addig a szimmetrikus korutinok esetén közvetlenül a korutin mondja meg, hogyan folytatódjon a program futása. Ekkor az ütemező mindössze a vezérlés átadásáért felel, a szimmetrikus korutin gondoskodik annak kijelöléséről, hogy ki kerüljön legközelebb ütemezésre.

A két eszköz tehát nem azonos célokat szolgál, mindössze nagyon hasonló az eszközkészlet, melynek segítségével megvalósíthatóak.

\section{Konklúzió}

A fejezet négy példán keresztül mutatta be a \textit{Jield} használatát és teljesítményét. Természetesen ez nem elég ahhoz, hogy általános érvényű következtetéseket vonhassunk le, azonban bizonyos jellemzők megfigyelését lehetővé teszik.

A \textit{Jield} segítségével ugyanúgy írhatunk generátort, mint egy hagyományos metódust, mindössze a \texttt{return} megváltozott szerepére kell tekintettel lenni. Ez lehetővé teszi, hogy közvetlenül az elemek előállítására koncentráljunk, a generátor működési mechanizmusa rejtve marad.

A generátorok valójában aszimmetrikus korutinok, így a \textit{Jield} által kínált generátorok is azok, a különleges tulajdonságokkal együtt. Ezekre a tulajdonságokra építve egészen izgalmas, új konstrukciót képezhetők, mint például a fejezetben is bemutatott kooperatív mikroszálak.

Az absztrakciónak azonban ára van. A mérésekből kitűnik, hogy legjobb esetben is kétszer, legrosszabb esetben mintegy $7,5$-szer több a \textit{Jield}-féle generátorok átlagos futási ideje, mint a \texttt{Stream}et használó metódusoké. Teljesítménykritikus helyzetekben ezt mindenképpen figyelembe kell venni.
