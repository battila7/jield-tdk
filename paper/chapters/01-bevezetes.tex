\chapter{Bevezetés}

\section{A dolgozat felépítése}

A dolgozatomban egy olyan transzformációs eljárást szeretnék ismertetni, mely lehetővé teszi generátorok előállítását, \textit{Java} nyelven. A módszer lépéseinek és megvalósításának részletezése előtt sor kerül az elméleti hátterét adó \textit{Continuation Passing Style} ismertetésére, valamint a generátorok elhelyezésére más, a programvezérlés útját befolyásoló szerkezetek között. Zárásként a bemutatott technika, és a generátorok \textit{Java}ban, jelenlegi eszközökkel való létrehozásának összehasonlítása szerepel, a teljesítményt, valamint az olvashatóságot alapul véve.

\section{Motiváció}

A \textit{Java} az objektumorientált paradigma mentén megtervezett, legutóbbi, \textit{Java 8} verziójában azonban már funkcionális jegyeket is magára öltő programozási nyelv \shortcite{JLS8}. A funkcionális paradigma irányába tett lépésként értékelhető például a \textit{Stream API} és a lambda függvények megjelenése \cite{Java8Features}. 

Bár ezek a fejlesztések jelentősek, azonban mondhatni folytatás nélkül maradnak, hiszen a 2017. július 27-én kiadásra kerülő \textit{Java 9} nem tartalmaz ilyen jellegű, az említettekkel megegyező horderejű újdonságokat \cite{Java9Features}. Tehát olyan eszközök támogatása, mint például a generátorok, sem a legközelebbi, sem a későbbi verziókban nem várható, a jelenlegi tervek alapján \cite{ProjectValhalla}.

Természetesen felmerül a kérdés, hogy mégis mennyiben indokolt egy ilyen, új nyelvi elem bevezetése? Egyelőre ahelyett, hogy a generátorok előnyeivel próbálnánk érvelni a bevezetésük mellett, vizsgáljuk meg inkább azt, hogy más, széleskörben használt programozási nyelvek hogyan viszonyulnak a kérdéshez. A legnépszerűbb nyelvek listájához a TIOBE Indexet vettem alapul \cite{TIOBEIndex}. Az \ref{table:1} táblázat adataiból kiolvasható, hogy a legnépszerűbb programozási nyelvek fele natívan támogatja a generátorok létrehozását. Ennek ismeretében már feltételezhetjük, hogy valós igény mutatkozna e struktúra bevezetésére a \textit{Java} nyelvben is. Az említett két tényező biztosította a motivációt, hogy a generátorok \textit{Java} nyelvben történő bevezetésének lehetőségét megvizsgáljam.

\if\useColors1
  \begin{table}
  \captionsetup{justification=centering}
  \centering
    \begin{tabular}{|| c | c ||}
    \hline
    Programozási nyelv & Támogatás \\
    \hline \hline
    Java                  & \cellcolor{red!20}Nem \\
    C                     & \cellcolor{red!20}Nem \\
    C++                   & \cellcolor{yellow!20}Tervezett \\
    C\#                   & \cellcolor{green!20}Igen \\
    Python                & \cellcolor{green!20}Igen \\
    Visual Basic .NET     & \cellcolor{green!20}Igen \\
    PHP                   & \cellcolor{green!20}Igen \\
    JavaScript            & \cellcolor{green!20}Igen \\
    Delphi/Object Pascal  & \cellcolor{red!20}Nem \\
    Swift                 & \cellcolor{red!20}Nem \\
    \hline
    \end{tabular}
  \caption{Generátorok támogatása a legnépszerűbb programozási nyelvek körében (népszerűség sorrendjében)}  
  \label{table:1}
  \end{table}
\else
  \begin{table}
  \captionsetup{justification=centering}
  \centering
    \begin{tabular}{|| c | c ||}
    \hline
    Programozási nyelv & Támogatás \\
    \hline \hline
    Java                  & Nem \\
    C                     & Nem \\
    C++                   & Tervezett \\
    C\#                   & Igen \\
    Python                & Igen \\
    Visual Basic .NET     & Igen \\
    PHP                   & Igen \\
    JavaScript            & Igen \\
    Delphi/Object Pascal  & Nem \\
    Swift                 & Nem \\
    \hline
    \end{tabular}
  \caption{Generátorok támogatása a legnépszerűbb programozási nyelvek körében (népszerűség sorrendjében)}  
  \label{table:1}
  \end{table}
\fi
\section{Az eljárás célja}

A dolgozatban ismertetett transzformációs eljárás azzal a céllal került kidolgozásra, hogy alapot vessen a natív támogatás megjelenésének. A kísérleti implementáció által szolgáltatott tapasztalatok birtokában ugyanis megnyílik az út a \textit{Java} fordító módosítása, egy új ajánlás létrehozása, és utolsó lépésként akár a hivatalos támogatás elnyerése előtt. 
Tehát a kidolgozott módszer és a dolgozat tekinthető egy hosszú eljárás első lépésének is, mely egy, a \textit{Java} fejlődését célzó \textit{Java Specification Request} kidolgozásában érhet véget. 

\section{Az implementációról}

Egy hasonló horderejű változtatás a nyelvben általában a fordító kiegészítését vonja maga után. Azonban a \textit{Java} által biztosított \citeA{JSR269} segítségével a nyelv és a fordító módosítása nélkül készíthetünk egy kísérleti implementációt. A dolgozat részeként elkészített könyvtár, a \textit{Jield} \footnote{játékos név, a \texttt{yield} kulcsszó és a \textit{Java} programkönyvtárak esetén gyakran használt \textit{J} előtag összevonásából} is az említett \textit{API} által kínált lehetőségeket aknázza ki. Ez különbözteti meg több hasonló megoldástól is, melyek vagy szálak alkalmazásán, vagy pedig bájtkód-instrumentáción alapulnak. Az annotáció-feldolgozás egy biztonságosabb, elegánsabb útja a generátorok bevezetésének. Létezik szintén az annotáció-feldolgozást használó megoldás is, mely azonban a bemutatásra kerülő \textit{CPS}-transzformációtól eltérő elven működik, és nem a nyelv ismert \texttt{return} utasítását használja \footnote{\url{https://github.com/peichhorn/lombok-pg/wiki/Yield}}.

Az elkészült implementáció kísérleti jellegű, azonban ilyen formában is alkalmat ad arra, hogy a létrehozott generátorokat összevessük értelmüket tekintve azonos, ám a generátorok eleganciáját és rugalmasságát nélkülöző kódrészletekkel. Az összehasonlítás részeként a különböző kódrészletek teljesítménye is megvizsgálásra kerül a \textit{JMH} (\textit{Java Microbenchmark Harness}) könyvtár felhasználásával.
