\chapter{A generátor és a vele rokon szerkezetek}

Mielőtt belekezdhetnénk a transzformáció lépéseinek és implementációjának tárgyalásába, meg kell ismernünk az alapfogalmakat, amivel az eljárás dolgozik. E fogalmak közül talán a legfontosabb maga a \textit{generátor}. A fejezet célja, hogy megmutassa, hogy mi az a generátor, valamint a teljesség igénye nélkül felvázolja a generátorral rokon szerkezetek családját. Ezzel egyfajta tágabb kontextusban tudjuk értelmezni a generátort.

A rokon szerkezetek között megtalálunk a programvezérlés szervezésére alkalmas struktúrákat, kifejezéseket, valamint adatszerkezetek bejárására szolgáló eszközöket.

\section{Szubrutin (\textit{subroutine})}

A szubrutin leggyakoribb felhasználása egy olyan kódrészlet kiemelése, melyet a programunk több részén is felhasználunk \cite{TAoCPVol1}. A kódduplikáció ilyen módon történő megszüntetése, illetve mérséklése a program méretét, és így a karbantartandó kód mennyiségét is csökkenti. 

A programrészletek szubrutinokba történő szervezése javítja a program olvashatóságát, érthetőbbé, követhetőbbé teszi a kódbázist. A szubrutin egyfajta absztrakciónak is tekinthető, mely elrejt egy utasítássorozatot, így az anélkül lesz felhasználható, hogy ismernénk a megvalósítás pontos részleteit. E módon a szubrutin hozzájárul a kódújrafelhasználáshoz is.

A \textit{refactoring} is előszeretettel alkalmazza a szubrutinokba szervezést. \textit{Martin Fowler} például \textit{Extract Method}nak nevezi a hosszú metódusok rövidebb, jobban átlátható egységekre való felbontását \cite{FowlerRefactoring}.

Érdeklődésünk középpontjában azonban a szubrutin programvezérlésre kifejtett hatása áll. Egy szubrutin meghívásakor a hívott és hívó között egy alá-fölérendeltségi viszony (\textit{subroutine linkage}) jön létre. A programvezérlés, miután a szubrutin törzse végrehajtásra került, visszatér a hívó kódhoz, mely csak ezt követően folytatódhat \cite{ScottProgLangPrag}. Természetesen egy szubrutin meghívhat más szubrutinokat is, azonban a vezérlés végül mindig vissza fog kerülni a szubrutint meghívó kódhoz.

\section{Korutin (\textit{coroutine})}

A \textit{korutin} fogalmának definiálása helyett, először kövessük Knuth gondolatmenetét \cite{TAoCPVol1}, melynek segítségével azt is megérthetjük, miért volt szükség a szubrutin ismertetésére.

Tekintsünk a főprogramra és a szubrutinokra együttesen úgy, mint programok egy csapatára, melyben minden egyes csapattag valamilyen feladatot végez el. A főprogram meghív egy szubrutint, az végrehajtja a megfelelő utasításokat, majd újra a főprogram következik. Tovább folytatva a gondolatsort, képzeljük el, hogy a szubrutin, amikor visszatér, valójában meghívja a főprogramot. 

Az előbbi gondolatokat követve, megkapjuk a korutinokat, amelyek esetében nincsen egyértelmű alá-fölérendeltségi viszony, mint a szubrutin és a főprogram esetén. A ko\-rutinok mind alá-fölérendeltségi, mind mellérendeltségi viszony kialakítására alkalmasak. Ennek megfelelően a szubrutinok valójában speciális korutinok \cite{TAoCPVol1}.

Felmerülhet azonban a kérdés, hogy mi történik abban az esetben, amikor egy korutin egynél többször kerül meghívásra? Míg szubrutinok hívásakor a végrehajtás mindig a szubrutin elejétől indul, addig korutinok esetén a végrehajtás mindig ott folytatódik, ahol a legutóbbi hívás esetén abbamaradt. Ez, és a korutin szempontjából lokális változók értékének megőrzése a hívások között, a korutinok két legfontosabb, definiáló tulajdonsága \cite{Marlin1980}.

\subsection{Szimmetrikus és aszimmetrikus korutinok}

A korutinok (vagyis azok megvalósításainak) különböző szempontok szerinti osztályozását adja meg de Moura és Ierusalimschy \citeyear{RevisitingCoroutines}. Egyik lehetséges osztályozási szempont a vezérlés átadásának módja, mely szerint szimmetrikus és aszimmetrikus korutinokat különböztethetünk meg.

A szimmetria valójában a korutinok egymás közötti viszonyát határozza meg. Ha a korutinok szimmetrikusak, akkor egy korutin tetszőleges másik korutint hívhat meg, felfüggesztve ezzel a saját futását, és aktívvá téve a másikat. Ezzel szemben az aszimmetria azt jelenti, hogy a hívott korutinnak vissza kell adnia a vezérlést a hívónak, felfüggesztve ezzel saját futását. Ez pontosan az a megvalósítás, amit a Knuth-féle gondolatmenet bevezetett. Ilyenkor, bár alárendeltségi viszony alakul ki, a Marlin-féle tulajdonságok továbbra is megmaradnak.

Az aszimmetrikus korutinokat elsőként Dahl vezette be, és félkorutinnak (\textit{semicoroutine}) nevezte őket \shortcite{StructuredProgramming}. Összegezve, a félkorutint többször is meghívhatjuk, mely hívások esetén mindig a megelőző visszatérés helye után fog folytatódni. A hívások között megőrzi lokális változóinak értékét, és képes felfüggeszteni a saját futását, így visszatérve a hívóhoz. Pontosan e tulajdonságokra van szükség a generátorok előállításához.

\section{Generátor (\textit{generator})}

A félkorutinok bevezetése után már a birtokunkban van egy alkalmas fogalom, mely felhasználható a generátorok kifejezéséhez. A generátor egy olyan félkorutin, amely elemek sorozatának előállítására képes. Ezt oly módon teszi, hogy minden egyes hívás alkalmával a sorozatnak egyetlen elemét képzi, majd visszaadva a vezérlést a hívónak, átadja a generált elemet. Pontosan ezt valósítja meg a \textit{CLU} nyelv \textit{iterator} szerkezete \cite{CLUManual}, melyet kifejezetten azzal a céllal vezettek be a nyelvbe, hogy lehetővé tegye a különböző adatszerkezetek elemeinek egyenkénti feldolgozását, egységes módon \cite{CLUHistory}.

Érdemes megemlíteni, hogy a \textit{CLU} \textit{iterator} szerkezetét az \textit{Alphard} generátorai inspirálták \cite{CLUHistory}, amelyek pedig az \textit{IPL-V} nyelv azonos nevű szerkezetében gyökereznek \shortcite{Shaw1976}. Marlin \citeyear{Marlin1980} szerint az \textit{IPL-V} nyelv generátorai jelentik a korutinok legkorábbi alkalmazását. A generátorok megvalósításában ma is használt \textit{yield} kulcsszó azonban csak a \textit{CLU}-ban került bevezetésre.

Természetesen a generátorok felhasználási területe nem csak az adatszerkezetek bejárására, feldolgozására korlátozódik. Használatukkal elemek tetszőleges sorozata előállítható, legyen az a sorozat véges, vagy végtelen. E működésre a generátorokat \textit{lusta} (\textit{lazy}) mivoltuk teszi alkalmassá, azaz, hogy csak akkor állítanak elő új elemet, ha szükséges.

\begin{lstlisting}[language=JavaScript, caption={Páratlan számok végtelen sorozatát előállító generátor JavaScriptben}, captionpos=b, label=JSGenOdds]
function *odds() {
    let i = 1;

    while (true) {
        yield i;

        i += 2;
    }
}
\end{lstlisting}

A \ref{JSGenOdds} kódrészlet egy \textit{JavaScript}ben írt, páratlan számok végtelen sorozatát előállító generátort ad meg. A \texttt{function} kulcsszó és a függvény neve között elhelyezett \texttt{*} karakter jelzi, hogy egy generátort definiálunk, melynek törzsében a \texttt{yield} kulcsszó segítségével tudjuk visszaadni a vezérlést a hívó kódnak, és átadni egy értéket. A példában, mivel a \texttt{yield} egy végtelen cikluson belül helyezkedik el, a generátor is végtelen sorozatot fog előállítani.

A \textit{CLU} a generátorok használatát csak a \texttt{for} ciklusokkal karöltve teszi lehetővé \cite{CLUManual}. Ennek az az oka, hogy a generátor sorozatos meghívását, így az új elemek előállítását a \texttt{for} szerkezet vezérli, nem a programozó. Ezzel szemben például a \textit{JavaScript} vagy a \textit{C\#} mindenhol lehetővé teszi a generátorok használatát, ahol hagyományos függvények is használhatóak. Ahhoz hogy megértsük, ez miért lehetséges, szükség van az iterátor fogalmára.

\section{Iterátor (\textit{iterator})}

A különböző adatszerkezetekben tárolt elemekhez gyakran egyesével is hozzá szeretnénk férni, hogy valamilyen műveletet végezzünk rajtuk. Ehhez ki kell nyernünk az adatszerkezetből az elemeket. Az \textit{iterátor} az adatszerkezet megvalósításának ismerete nélkül teszi lehetővé a tárolt elemeken való iterációt \cite{Baker1992}. Az elemek kinyerésének logikáját e módon elválasztja az elemeken végzett művelet logikájától.

Ahhoz, hogy az iterátor képes legyen az adatszerkezet bejárására, ismernie kell annak belső megvalósítását. \textit{C++}-ban ezt általában az osztályok közötti \textit{friend} kapcsolat segítségével oldják meg \shortcite{GammaDesignPatterns}. \textit{Java}ban az iterátor az adatszerkezet egy \textit{nem statikus tagosztályaként} (\textit{inner class}) kerülhet definiálásra ugyanilyen helyzetben (lsd.: \texttt{java.util.ArrayList} vagy \texttt{java.util.HashMap}). Az implementáció ismerete mellett az iterátoroknak \textit{állapotra} van szükségük ahhoz, hogy a bejárást folytatni tudják. 

\subsection{Belső és külső iterátor (\textit{internal and external iterator})}

Az iterátor egy olyan minta, amely attól függően, hogy milyen tulajdonságokat várunk el tőle, rendkívül sokféle módon implementálható. Az egyik szempont, ami befolyásolhatja a megvalósítást, az iteráció vezérlésének kérdése. Aszerint, hogy ki vezérli az iterációt, kétféle iterátor-megvalósítást különböztetünk meg.

A \textit{belső} vagy \textit{passzív iterátor} saját maga vezérli az elemekhez való hozzáférést. A külső, kliens kódnak csak az elemeken elvégzendő műveletet kell szolgáltatnia. Ennek a módszernek az előnye, hogy a bejárást vezérlő kódot csak egyszer, az iterátor megvalósításakor kell megírni, a kliens kódban már nem szükséges \cite{Martin1994}. A belső iterátorokat tipikusan az adatszerkezeteket megvalósító osztályok bocsájtják a kliensek rendelkezésére \cite{Kuhne1999}.

\begin{lstlisting}[language=JavaScript, caption={Tömb elemeinek kilistázása belső iterátorral JavaScriptben}, captionpos=b, label=JSInternalIt]
const numbers = [1, 2, 3, 4, 5];

numbers.forEach(function print(element) {
    console.log(element);
});
\end{lstlisting}

A \ref{JSInternalIt} kódrészlet a \textit{JavaScript} \texttt{Array} objektuma által biztosított belső iterátor segítségével listázza ki a \texttt{numbers} tömb elemeit. Vegyük észre, hogy csak az elemeken végrehajtandó műveletet kellett megadni egy függvény formájában. Az adatszerkezet megvalósítása és bejárásának logikája rejtve marad.

\textit{Külső} vagy \textit{aktív iterátor} használatakor a kliensre hárul az iteráció vezérlésének feladata. Az iterátor biztosítja az ehhez szükséges eszközöket, mint például a következő elemre lépés, vagy az aktuális elem lekérdezése. Ez a megoldás, bár több terhet ró a kliensre, rugalmasabb, mint a belső iterátor \shortcite{GammaDesignPatterns}. Erre példa az iteráció idő előtt való befejezése (azaz a teljes adatszerkezet bejárásának megelőzése), mely külső iterátor használatával triviális. 

\begin{lstlisting}[language=JavaScript, caption={Tömb elemeinek kilistázása külső iterátorral JavaScriptben}, captionpos=b, label=JSExternalIt]
const numbers = [1, 2, 3, 4, 5];

const it = numbers[Symbol.iterator]();

let state = it.next();

while (!state.done) {
    console.log(state.value);

    state = it.next();
}
\end{lstlisting}

A \ref{JSExternalIt} kódrészlet ugyanúgy az elemek kiíratását valósítja meg, mint a \ref{JSInternalIt} kódrészlet, azonban külső iterátor alkalmazásával. A következő elemet mindig a \texttt{next()} függvény meghívásával állítottuk elő. Feltűnő, hogy az iteráció vezérlése mekkora kódmennyiséget jelent a tényleges műveleten felül.

A külső és belső iterátorok részletes összehasonlítása megtalálható Thomas Kühne \textit{Internal Iteration Externalized} című munkájában.

\subsection{Iterálható (\textit{iterable})}

A \textit{Java} és a \textit{Python} az iterátoron felül egy további absztrakciót biztosít, az \textit{iterálhatót}. Iterálható lehet bármilyen olyan osztály, amely képes valamilyen külső iterátort biztosítani a kliens számára; tipikusan ilyenek az elemek gyűjteményei, mint a lista vagy a halmaz \cite{nvieitvsgen}. Az iterálható kihangsúlyozza az iterátor és az iterált adatszerkezet közötti különbséget.

\subsection{Generátorok megvalósítása iterátorokkal}

Az iterátorok viszonylag részletes bemutatása meglepő lehet abból a szempontból, hogy rokonságuk a korutinokkal távolinak tűnik. Az összekötő kapocs nem lesz más, mint a generátor. A generátorok, ha a programvezérlés szempontjából tekintünk rájuk, akkor a speciális félkorutin mivoltukat fedik fel. Ha viszont arra gondolunk, hogy egy sorozat elemeit képesek egyenként előállítani, akkor beláthatjuk, hogy egyúttal speciális iterátornak is tekinthetők.

Ez a rokonság az alapja annak, hogy a korutinokat nem, de iterátorokat ismerő programozási nyelvekben is dolgozhassunk generátorokkal. Az egyik lehetőség, hogy a függvényeket \textit{iterálhatónak} tekintjük. Ebben az esetben a függvény meghívásával egy új \textit{iterátort} kapunk, mely a generátor által lustán előállított elemeken képes iterálni. Ezt alkalmazza például a \textit{JavaScript}. A másik lehetőség, hogy a függvény egy \textit{iterálhatóval} tér vissza, melyet ugyanúgy használhatunk, mint egy iterálható gyűjteményt. E megoldás alkalmazására példa a \textit{C\#}.

A függvény eredeti törzsét azonban nem használhatjuk fel közvetlenül, hiszen ko\-rutinok hiányában nem rendelkezünk olyan eszközökkel, melyek lehetővé tennék a lokális változók értékének hívások közötti megőrzését és a végrehajtás folytatását. A függvénytörzset fordítási idejű transzformációnak kell alávetnünk, hogy ezeket a tulajdonságokat szimulálni tudjuk. A dolgozatban ismertetésre kerülő transzformáció célja is pontosan ugyanez.

% Lehet, hogy el kell majd távolítani
% \pagebreak

\section{Generátor-kifejezés (\textit{generator expression})}

A szubrutinoktól az iterátorokig végigvezetett gondolatmenet ívét megtörte volna a következő eszköz, mely azonban számot tarthat az olvasó érdeklődésére, hiszen szorosan kapcsolódik a generátorokhoz. 

Néhány esetben fárasztó lehet generátort írni, vagy nem is ismeri ezt a fogalmat a programozási nyelv, amiben dolgozunk. Mindkét esetben megoldást kínálhatnak a \textit{generátor kifejezések} (bizonyos nyelvekben \textit{comprehension}), melyek megtartják a generátorok legnagyobb előnyét (az elemek lusta előállítását), de minimalizálják a szükséges kódmennyiséget. 

Szemléletes példa a \textit{Python}, mely bár ismeri a generátorokat, lehetővé teszi generátor-kifejezések használatát is \cite{PEP289}.

\begin{center}
\begin{minipage}[t]{0.45\textwidth}
\begin{lstlisting}[language=Python]
g = (x**2 for x in range(10))

print g.next()
\end{lstlisting}
\end{minipage} 
\begin{minipage}[t]{0.45\textwidth}
\begin{lstlisting}[language=Python,]
def __gen(exp):
    for x in exp:
        yield x**2

g = __gen(iter(range(10)))

print g.next()
\end{lstlisting} 
\end{minipage}
\captionof{lstlisting}{Négyzetszámok előállítása generátor-kifejezés és generátor segítségével Pythonban}
\label{PythonGenComparison}
\end{center}

A \ref{PythonGenComparison} kódrészlet jobb- és baloldala is pontosan ugyanazt valósítja meg, azonban a baloldali, generátor-kifejezést használó kód jóval tömörebb formában.

A \textit{Java}val közelebbi rokonságban álló nyelvek közül a \textit{Ceylon}t lehet kiemelni, mely \textit{comprehension} néven biztosít hasonló lehetőséget \cite{CeylonComprehension}. Fontos megjegyezni, hogy \textit{Ceylon}ban nincsen nyelvi szintű generátor, szemben a \textit{Python}nal.

\begin{lstlisting}[language=Ceylon, caption={Nevek kilistázása generátor-kifejezéssel Ceylonban}, captionpos=b, label=CeylonGenExpr]
class Person(shared String name) {}

value people = { Person("Gavin"), Person("Stephane"), Person("Tom") };
          
{String*} names = { for (p in people) p.name };
          
print(names);
\end{lstlisting}

A \ref{CeylonGenExpr} kódrészlet ötödik sorában található generátor-kifejezés neveket állít elő, azonban az értékadás pillanatában a kifejezés még nem generál egy nevet sem. Ez csak akkor fog bekövetkezni, amikor ténylegesen fel kívánjuk használni a generált elemeket, például ki szeretnénk listázni őket. 