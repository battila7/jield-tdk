% Különleges karakterek használatának lehetősége kódrészletben
\lstset{literate=
 {á}{{\'a}}1 {é}{{\'e}}1 {í}{{\'i}}1 {ó}{{\'o}}1 {ú}{{\'u}}1
 {Á}{{\'A}}1 {É}{{\'E}}1 {Í}{{\'I}}1 {Ó}{{\'O}}1 {Ú}{{\'U}}1
 {ö}{{\"o}}1 {ü}{{\"u}}1 {Ö}{{\"O}}1 {Ü}{{\"U}}1
 {ű}{{\H{u}}}1 {Ű}{{\H{U}}}1 {ő}{{\H{o}}}1 {Ő}{{\H{O}}}1
}

% A kódrészletek beállításai
\definecolor{codegray}{rgb}{0.5,0.5,0.5}
\lstset{xleftmargin=15pt,
        basicstyle=\scriptsize,
        numbers=left,
        numbersep=5pt,
        numberstyle=\tiny\color{codegray},
        escapechar=!,
        aboveskip=2em,
        belowskip=2em,
        belowcaptionskip=2em}

% Fix listings
\renewcommand{\lstlistingname}{Kódrészlet}

% Fix APA
\renewcommand\BOthers{és mtsai\hbox{}}
\renewcommand\BOthersPeriod{és mtsai.\hbox{}}
\renewcommand\BRetrievedFrom{Letöltve:\ }
\renewcommand\BRetrieved[1]{Letöltve, {#1}:\ }

% Elválasztás
\hyphenation{ko-ru-tin}

