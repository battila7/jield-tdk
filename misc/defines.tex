% Különleges karakterek használatának lehetősége kódrészletben
\lstset{literate=
 {á}{{\'a}}1 {é}{{\'e}}1 {í}{{\'i}}1 {ó}{{\'o}}1 {ú}{{\'u}}1
 {Á}{{\'A}}1 {É}{{\'E}}1 {Í}{{\'I}}1 {Ó}{{\'O}}1 {Ú}{{\'U}}1
 {ö}{{\"o}}1 {ü}{{\"u}}1 {Ö}{{\"O}}1 {Ü}{{\"U}}1
 {ű}{{\H{u}}}1 {Ű}{{\H{U}}}1 {ő}{{\H{o}}}1 {Ő}{{\H{O}}}1
}

% A kódrészletek beállításai
\definecolor{codegray}{rgb}{0.5,0.5,0.5}
\lstset{xleftmargin=15pt,
        basicstyle=\scriptsize,
        numbers=left,
        numbersep=5pt,
        numberstyle=\tiny\color{codegray},
        escapechar=@,
        aboveskip=2em,
        belowskip=2em,
        belowcaptionskip=2em}

% JS nyelv
\lstdefinelanguage{JavaScript}{
  keywords={typeof, new, true, false, catch, function, return, null, catch, switch, var, if, in, while, do, else, case, break, yield, let, const        },
  keywordstyle=\bfseries,
  ndkeywords={class, export, boolean, throw, implements, import, this},
  ndkeywordstyle=\bfseries,
  identifierstyle=\color{black},
  sensitive=false,
  comment=[l]{//},
  morecomment=[s]{/*}{*/},
  commentstyle=\ttfamily,
  stringstyle=\ttfamily,
  morestring=[b]',
  morestring=[b]"
}

% Ceylon nyelv
\lstdefinelanguage{Ceylon}{
  keywords={value, String, for, in, class, shared},
  keywordstyle=\bfseries,
  ndkeywordstyle=\bfseries,
  identifierstyle=\color{black},
  sensitive=false,
  comment=[l]{//},
  morecomment=[s]{/*}{*/},
  commentstyle=\ttfamily,
  stringstyle=\ttfamily,
  morestring=[b]',
  morestring=[b]"
}

% Fix listings
\renewcommand{\lstlistingname}{Kódrészlet}

% Fix APA
\renewcommand\BOthers{és mtsai\hbox{}}
\renewcommand\BOthersPeriod{és mtsai.\hbox{}}
\renewcommand\BRetrievedFrom{Letöltve:\ }
\renewcommand\BRetrieved[1]{Letöltve, {#1}:\ }
\renewcommand\BIn{}
\renewcommand\BED{Szerk. \hbox{}}
\renewcommand\BEDS{Szerk. \hbox{}}

% Elválasztás
\hyphenation{ko-ru-tin}

\renewcommand{\ellipsisgap}{0.1em}

% 1,5-es sorköz
\linespread{1.25}