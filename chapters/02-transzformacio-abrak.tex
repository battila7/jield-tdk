\chapter*{WIP - Transzformáció ábrák}

A transzformáció lépéseinek leírásához kerülő ábrák.

% --- Lineáris ---
\begin{center}
\begin{mdframed}
\begin{minipage}[t]{0.4\textwidth}
\begin{lstlisting}[language=Java, numbers=none, breaklines=true]
! \tikz[remember picture, overlay]\node[rounded corners, draw, xshift=-0.1cm, inner sep=5pt, anchor=west] {Kódrészlet \#1}; \vspace*{0.5cm} ! 
return ! \tikz[remember picture, overlay]\node[rounded corners, draw, xshift=-0.1cm, inner sep=5pt, anchor=west, yshift=0.1cm] {Kifejezés \#1}; \vspace*{0.3cm} !
! \tikz[remember picture, overlay]\node[rounded corners, draw, xshift=-0.1cm, inner sep=5pt, anchor=west] {Kódrészlet \#2}; \vspace*{0.5cm} ! 
return ! \tikz[remember picture, overlay]\node[rounded corners, draw, xshift=-0.1cm, inner sep=5pt, anchor=west, yshift=0.1cm] {Kifejezés \#2}; \vspace*{0.3cm} !
\end{lstlisting}
\end{minipage} 
\begin{minipage}[t]{0.6\textwidth}
\begin{lstlisting}[language=Java, numbers=none, breaklines=true]
Bounce<T> method1(GeneratorState<T> k) {
  ! \tikz[remember picture, overlay]\node[rounded corners, draw, xshift=-0.1cm, inner sep=5pt, anchor=west] {Kódrészlet \#1}; \vspace*{0.3cm} !
  return cont(() -> method2(k), ! \tikz[remember picture, overlay]\node[rounded corners, draw, xshift=-0.1cm, inner sep=5pt, anchor=west, yshift=0.1cm] {Kifejezés \#1}; \vspace*{0.2cm} \hspace*{1.75cm} !);
}

Bounce<T> method2(GeneratorState<T> k) {
  ! \tikz[remember picture, overlay]\node[rounded corners, draw, xshift=-0.1cm, inner sep=5pt, anchor=west] {Kódrészlet \#2}; \vspace*{0.3cm} !
  return cont(() -> k.apply(null), ! \tikz[remember picture, overlay]\node[rounded corners, draw, xshift=-0.2cm, inner sep=5pt, anchor=west, yshift=0.1cm] {Kifejezés \#2}; \vspace*{0.2cm} \hspace*{1.6cm} !);
}
\end{lstlisting} 
\end{minipage}
\end{mdframed}
\captionof{lstlisting}{Lineáris vezérlésű kód transzformálása}
\end{center}

\pagebreak

% --- for ciklus ---
\begin{center}
\begin{mdframed}[topline=true]
\begin{minipage}[t]{0.4\textwidth}
\begin{lstlisting}[language=Java, numbers=none, breaklines=true]
for (! \tikz[remember picture, overlay]\node[rounded corners, draw, xshift=-0.1cm, inner sep=5pt, anchor=west, yshift=0.1cm] {Init}; \hspace*{0.6cm}!; ! \tikz[remember picture, overlay]\node[rounded corners, draw, xshift=-0.2cm, inner sep=5pt, anchor=west, yshift=0.1cm] {Feltétel}; \hspace*{1.0cm}!; ! \tikz[remember picture, overlay]\node[rounded corners, draw, xshift=-0.2cm, inner sep=5pt, anchor=west, yshift=0.1cm] {Frissít}; \vspace*{0.3cm} \hspace*{0.9cm}!) {
  ! \tikz[remember picture, overlay]\node[rounded corners, draw, xshift=-0.1cm, inner sep=5pt, anchor=west] {Kódrészlet}; \vspace*{0.5cm} ! 
  return ! \tikz[remember picture, overlay]\node[rounded corners, draw, xshift=-0.1cm, inner sep=5pt, anchor=west, yshift=0.1cm] {Kifejezés}; \vspace*{0.3cm} !
}
\end{lstlisting}
\end{minipage} 
\begin{minipage}[t]{0.6\textwidth}
\begin{lstlisting}[language=Java, numbers=none, breaklines=true]
Bounce<T> method1(GeneratorState<T> k) {
  ! \tikz[remember picture, overlay]\node[rounded corners, draw, xshift=-0.1cm, inner sep=5pt, anchor=west] {Init}; \vspace*{0.3cm} !
  return cont(() -> method2(k));
}

Bounce<T> method2(GeneratorState<T> k) {
  if (! \tikz[remember picture, overlay]\node[rounded corners, draw, xshift=-0.1cm, inner sep=5pt, anchor=west, yshift=0.1cm] {Feltétel}; \hspace*{1.04cm} !) {
    return cont(() -> method3(k));
  }
  return cont(() -> k.apply(null));
}

Bounce<T> method3(GeneratorState<T> k) {
  ! \tikz[remember picture, overlay]\node[rounded corners, draw, xshift=-0.1cm, inner sep=5pt, anchor=west] {Kódrészlet}; \vspace*{0.3cm} !
  return cont(() -> method4(k), ! \tikz[remember picture, overlay]\node[rounded corners, draw, xshift=-0.1cm, inner sep=5pt, anchor=west, yshift=0.1cm] {Kifejezés}; \vspace*{0.2cm} \hspace*{1.2cm} !);
}

Bounce<T> method4(GeneratorState<T> k) {
  ! \tikz[remember picture, overlay]\node[rounded corners, draw, xshift=-0.1cm, inner sep=5pt, anchor=west] {Frissít}; \vspace*{0.3cm} !
  return cont(() -> method2(k));
}
\end{lstlisting} 
\end{minipage}
\end{mdframed}
\captionof{lstlisting}{\texttt{for} ciklus transzformálása}
\end{center}

\pagebreak

% --- while ciklus ---
\begin{center}
\begin{mdframed}[topline=true]
\begin{minipage}[t]{0.4\textwidth}
\begin{lstlisting}[language=Java, numbers=none, breaklines=true]
while (! \tikz[remember picture, overlay]\node[rounded corners, draw, xshift=-0.1cm, inner sep=5pt, anchor=west, yshift=0.1cm] {Feltétel}; \vspace*{0.3cm} \hspace*{1.14cm}!) {
  ! \tikz[remember picture, overlay]\node[rounded corners, draw, xshift=-0.1cm, inner sep=5pt, anchor=west] {Kódrészlet}; \vspace*{0.5cm} ! 
  return ! \tikz[remember picture, overlay]\node[rounded corners, draw, xshift=-0.1cm, inner sep=5pt, anchor=west, yshift=0.1cm] {Kifejezés}; \vspace*{0.3cm} !
}
\end{lstlisting}
\end{minipage} 
\begin{minipage}[t]{0.6\textwidth}
\begin{lstlisting}[language=Java, numbers=none, breaklines=true]
Bounce<T> method1(GeneratorState<T> k) {
  if (! \tikz[remember picture, overlay]\node[rounded corners, draw, xshift=-0.1cm, inner sep=5pt, anchor=west, yshift=0.1cm] {Feltétel}; \hspace*{1.04cm} !) {
    return cont(() -> method2(k));
  }
  return cont(() -> k.apply(null));
}

Bounce<T> method2(GeneratorState<T> k) {
  ! \tikz[remember picture, overlay]\node[rounded corners, draw, xshift=-0.1cm, inner sep=5pt, anchor=west] {Kódrészlet}; \vspace*{0.3cm} !
  return cont(() -> method1(k), ! \tikz[remember picture, overlay]\node[rounded corners, draw, xshift=-0.1cm, inner sep=5pt, anchor=west, yshift=0.1cm] {Kifejezés}; \vspace*{0.2cm} \hspace*{1.2cm} !);
}
\end{lstlisting} 
\end{minipage}
\end{mdframed}
\captionof{lstlisting}{\texttt{while} ciklus transzformálása}
\end{center}

% --- do-while ciklus ---
\begin{center}
\begin{mdframed}[topline=true]
\begin{minipage}[t]{0.4\textwidth}
\begin{lstlisting}[language=Java, numbers=none, breaklines=true]
do {
  ! \tikz[remember picture, overlay]\node[rounded corners, draw, xshift=-0.1cm, inner sep=5pt, anchor=west] {Kódrészlet}; \vspace*{0.5cm} ! 
  return ! \tikz[remember picture, overlay]\node[rounded corners, draw, xshift=-0.1cm, inner sep=5pt, anchor=west, yshift=0.1cm] {Kifejezés}; \vspace*{0.3cm} !
} while (! \tikz[remember picture, overlay]\node[rounded corners, draw, xshift=-0.1cm, inner sep=5pt, anchor=west, yshift=0.1cm] {Feltétel}; \vspace*{0.3cm} \hspace*{1.14cm}!);
\end{lstlisting}
\end{minipage} 
\begin{minipage}[t]{0.6\textwidth}
\begin{lstlisting}[language=Java, numbers=none, breaklines=true]
Bounce<T> method1(GeneratorState<T> k) {
  ! \tikz[remember picture, overlay]\node[rounded corners, draw, xshift=-0.1cm, inner sep=5pt, anchor=west] {Kódrészlet}; \vspace*{0.3cm} !
  return cont(() -> method2(k), ! \tikz[remember picture, overlay]\node[rounded corners, draw, xshift=-0.1cm, inner sep=5pt, anchor=west, yshift=0.1cm] {Kifejezés}; \vspace*{0.2cm} \hspace*{1.2cm} !);
}

Bounce<T> method2(GeneratorState<T> k) {
  if (! \tikz[remember picture, overlay]\node[rounded corners, draw, xshift=-0.1cm, inner sep=5pt, anchor=west, yshift=0.1cm] {Feltétel}; \hspace*{1.04cm} !) {
    return cont(() -> method1(k));
  }
  return cont(() -> k.apply(null));
}
\end{lstlisting} 
\end{minipage}
\end{mdframed}
\captionof{lstlisting}{\texttt{do-while} ciklus transzformálása}
\end{center}


\pagebreak

% --- foreach ciklus ---
\begin{center}
\begin{mdframed}[topline=true]
\begin{minipage}[t]{0.4\textwidth}
\begin{lstlisting}[language=Java, numbers=none, breaklines=true]
for (! \tikz[remember picture, overlay]\node[rounded corners, draw, xshift=-0.1cm, inner sep=5pt, anchor=west, yshift=0.1cm] {Változó}; \hspace*{1.05cm}! : ! \tikz[remember picture, overlay]\node[rounded corners, draw, xshift=-0.2cm, inner sep=5pt, anchor=west, yshift=0.1cm] {Forrás}; \vspace*{0.3cm} \hspace*{0.9cm}!) {
  ! \tikz[remember picture, overlay]\node[rounded corners, draw, xshift=-0.1cm, inner sep=5pt, anchor=west] {Kódrészlet}; \vspace*{0.5cm} ! 
  return ! \tikz[remember picture, overlay]\node[rounded corners, draw, xshift=-0.1cm, inner sep=5pt, anchor=west, yshift=0.1cm] {Kifejezés}; \vspace*{0.3cm} !
}
\end{lstlisting}
\end{minipage} 
\begin{minipage}[t]{0.6\textwidth}
\begin{lstlisting}[language=Java, numbers=none, breaklines=true]
Bounce<T> method1(GeneratorState<T> k) { !\vspace{0.2cm}!
  iterator = CPSUtil.iterator(! \tikz[remember picture, overlay]\node[rounded corners, draw, xshift=-0.1cm, yshift=0.1cm, inner sep=5pt, anchor=west] {Forrás}; \vspace{0.3cm} \hspace*{1.0cm}!);
  return cont(() -> method2(k));
}

Bounce<T> method2(GeneratorState<T> k) {
  if (iterator.hasNext()) { !\vspace{0.2cm}!
    ! \tikz[remember picture, overlay]\node[rounded corners, draw, xshift=-0.1cm, inner sep=5pt, anchor=west, yshift=0.1cm] {Változó}; \vspace*{0.2cm} \hspace*{1.1cm}! = iterator.next();
    return cont(() -> method3(k));
  }
  return cont(() -> k.apply(null));
}

Bounce<T> method3(GeneratorState<T> k) {
  ! \tikz[remember picture, overlay]\node[rounded corners, draw, xshift=-0.1cm, inner sep=5pt, anchor=west] {Kódrészlet}; \vspace*{0.3cm} !
  return cont(() -> method2(k), ! \tikz[remember picture, overlay]\node[rounded corners, draw, xshift=-0.1cm, inner sep=5pt, anchor=west, yshift=0.1cm] {Kifejezés}; \vspace*{0.2cm} \hspace*{1.2cm} !);
}
\end{lstlisting} 
\end{minipage}
\end{mdframed}
\captionof{lstlisting}{\texttt{foreach} ciklus transzformálása}
\end{center}

\pagebreak

% --- switch ---
\begin{center}
\begin{mdframed}[topline=true]
\begin{minipage}[t]{0.4\textwidth}
\begin{lstlisting}[language=Java, numbers=none, breaklines=true]
switch (! \tikz[remember picture, overlay]\node[rounded corners, draw, xshift=-0.1cm, inner sep=5pt, anchor=west, yshift=0.1cm] {Szelektor}; \vspace*{0.3cm} \hspace*{1.35cm}!) {
  case ! \tikz[remember picture, overlay]\node[rounded corners, draw, xshift=-0.1cm, inner sep=5pt, anchor=west, yshift=0.1cm] {Konstans}; \vspace*{0.3cm} \hspace*{1.35cm}!:
    ! \tikz[remember picture, overlay]\node[rounded corners, draw, xshift=-0.1cm, inner sep=5pt, anchor=west] {Kódrészlet \#1}; \vspace*{0.5cm} ! 
    return ! \tikz[remember picture, overlay]\node[rounded corners, draw, xshift=-0.1cm, inner sep=5pt, anchor=west, yshift=0.1cm] {Kifejezés \#1}; \vspace*{0.3cm} !

  default :
    ! \tikz[remember picture, overlay]\node[rounded corners, draw, xshift=-0.1cm, inner sep=5pt, anchor=west] {Kódrészlet \#2}; \vspace*{0.5cm} ! 
    return ! \tikz[remember picture, overlay]\node[rounded corners, draw, xshift=-0.1cm, inner sep=5pt, anchor=west, yshift=0.1cm] {Kifejezés \#2}; \vspace*{0.3cm} !
}
\end{lstlisting}
\end{minipage} 
\begin{minipage}[t]{0.6\textwidth}
\begin{lstlisting}[language=Java, numbers=none, breaklines=true]
Bounce<T> method1(GeneratorState<T> k) { !\vspace{0.2cm}!
  switch (! \tikz[remember picture, overlay]\node[rounded corners, draw, xshift=-0.1cm, inner sep=5pt, anchor=west, yshift=0.1cm] {Szelektor}; \vspace*{0.3cm} \hspace*{1.35cm}!) {
    case ! \tikz[remember picture, overlay]\node[rounded corners, draw, xshift=-0.1cm, inner sep=5pt, anchor=west, yshift=0.1cm] {Konstans}; \vspace*{0.1cm} \hspace*{1.35cm}!:
      return cont(() -> method2(k));

    default :
      return cont(() -> method3(k));
  }
}

Bounce<T> method2(GeneratorState<T> k) {
  ! \tikz[remember picture, overlay]\node[rounded corners, draw, xshift=-0.1cm, inner sep=5pt, anchor=west] {Kódrészlet \#1}; \vspace*{0.3cm} !
  return cont(() -> method4(k), ! \tikz[remember picture, overlay]\node[rounded corners, draw, xshift=-0.1cm, inner sep=5pt, anchor=west, yshift=0.1cm] {Kifejezés \#1}; \vspace*{0.2cm} \hspace*{1.75cm} !);
}

Bounce<T> method3(GeneratorState<T> k) {
  ! \tikz[remember picture, overlay]\node[rounded corners, draw, xshift=-0.1cm, inner sep=5pt, anchor=west] {Kódrészlet \#2}; \vspace*{0.3cm} !
  return cont(() -> method4(k), ! \tikz[remember picture, overlay]\node[rounded corners, draw, xshift=-0.1cm, inner sep=5pt, anchor=west, yshift=0.1cm] {Kifejezés \#2}; \vspace*{0.2cm} \hspace*{1.75cm} !);
}

Bounce<T> method4(GeneratorState<T> k) {
  return cont(() -> k.apply(null));
}
\end{lstlisting} 
\end{minipage}
\end{mdframed}
\captionof{lstlisting}{\texttt{switch} utasítás transzformálása}
\end{center}
