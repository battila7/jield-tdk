\chapter*{WIP - Transzformáció ábrák}

A transzformáció lépéseinek leírásához kerülő ábrák.

% --- while ciklus ---
\begin{center}
\begin{mdframed}[topline=true]
\begin{minipage}[t]{0.4\textwidth}
\begin{lstlisting}[language=Java, numbers=none, breaklines=true]
while (@ \tikz[remember picture, overlay]\node[rounded corners, draw, xshift=-0.1cm, inner sep=5pt, anchor=west, yshift=0.1cm] {Feltétel}; \vspace*{0.3cm} \hspace*{1.14cm}@) {
  @ \tikz[remember picture, overlay]\node[rounded corners, draw, xshift=-0.1cm, inner sep=5pt, anchor=west] {Kódrészlet}; \vspace*{0.5cm} @ 
  return @ \tikz[remember picture, overlay]\node[rounded corners, draw, xshift=-0.1cm, inner sep=5pt, anchor=west, yshift=0.1cm] {Kifejezés}; \vspace*{0.3cm} @
}
\end{lstlisting}
\end{minipage} 
\begin{minipage}[t]{0.6\textwidth}
\begin{lstlisting}[language=Java, numbers=none, breaklines=true]
Bounce<T> method1(GeneratorState<T> k) {
  if (@ \tikz[remember picture, overlay]\node[rounded corners, draw, xshift=-0.1cm, inner sep=5pt, anchor=west, yshift=0.1cm] {Feltétel}; \hspace*{1.04cm} @) {
    return cont(() -> method2(k));
  }
  return cont(() -> k.apply(null));
}

Bounce<T> method2(GeneratorState<T> k) {
  @ \tikz[remember picture, overlay]\node[rounded corners, draw, xshift=-0.1cm, inner sep=5pt, anchor=west] {Kódrészlet}; \vspace*{0.3cm} @
  return cont(() -> method1(k), @ \tikz[remember picture, overlay]\node[rounded corners, draw, xshift=-0.1cm, inner sep=5pt, anchor=west, yshift=0.1cm] {Kifejezés}; \vspace*{0.2cm} \hspace*{1.2cm} @);
}
\end{lstlisting} 
\end{minipage}
\end{mdframed}
\captionof{lstlisting}{\texttt{while} ciklus transzformálása}
\end{center}

% --- do-while ciklus ---
\begin{center}
\begin{mdframed}[topline=true]
\begin{minipage}[t]{0.4\textwidth}
\begin{lstlisting}[language=Java, numbers=none, breaklines=true]
do {
  @ \tikz[remember picture, overlay]\node[rounded corners, draw, xshift=-0.1cm, inner sep=5pt, anchor=west] {Kódrészlet}; \vspace*{0.5cm} @ 
  return @ \tikz[remember picture, overlay]\node[rounded corners, draw, xshift=-0.1cm, inner sep=5pt, anchor=west, yshift=0.1cm] {Kifejezés}; \vspace*{0.3cm} @
} while (@ \tikz[remember picture, overlay]\node[rounded corners, draw, xshift=-0.1cm, inner sep=5pt, anchor=west, yshift=0.1cm] {Feltétel}; \vspace*{0.3cm} \hspace*{1.14cm}@);
\end{lstlisting}
\end{minipage} 
\begin{minipage}[t]{0.6\textwidth}
\begin{lstlisting}[language=Java, numbers=none, breaklines=true]
Bounce<T> method1(GeneratorState<T> k) {
  @ \tikz[remember picture, overlay]\node[rounded corners, draw, xshift=-0.1cm, inner sep=5pt, anchor=west] {Kódrészlet}; \vspace*{0.3cm} @
  return cont(() -> method2(k), @ \tikz[remember picture, overlay]\node[rounded corners, draw, xshift=-0.1cm, inner sep=5pt, anchor=west, yshift=0.1cm] {Kifejezés}; \vspace*{0.2cm} \hspace*{1.2cm} @);
}

Bounce<T> method2(GeneratorState<T> k) {
  if (@ \tikz[remember picture, overlay]\node[rounded corners, draw, xshift=-0.1cm, inner sep=5pt, anchor=west, yshift=0.1cm] {Feltétel}; \hspace*{1.04cm} @) {
    return cont(() -> method1(k));
  }
  return cont(() -> k.apply(null));
}
\end{lstlisting} 
\end{minipage}
\end{mdframed}
\captionof{lstlisting}{\texttt{do-while} ciklus transzformálása}
\end{center}


\pagebreak

% --- switch ---
\begin{center}
\begin{mdframed}[topline=true]
\begin{minipage}[t]{0.4\textwidth}
\begin{lstlisting}[language=Java, numbers=none, breaklines=true]
switch (@ \tikz[remember picture, overlay]\node[rounded corners, draw, xshift=-0.1cm, inner sep=5pt, anchor=west, yshift=0.1cm] {Szelektor}; \vspace*{0.3cm} \hspace*{1.35cm}@) {
  case @ \tikz[remember picture, overlay]\node[rounded corners, draw, xshift=-0.1cm, inner sep=5pt, anchor=west, yshift=0.1cm] {Konstans}; \vspace*{0.3cm} \hspace*{1.35cm}@:
    @ \tikz[remember picture, overlay]\node[rounded corners, draw, xshift=-0.1cm, inner sep=5pt, anchor=west] {Kódrészlet \#1}; \vspace*{0.5cm} @ 
    return @ \tikz[remember picture, overlay]\node[rounded corners, draw, xshift=-0.1cm, inner sep=5pt, anchor=west, yshift=0.1cm] {Kifejezés \#1}; \vspace*{0.3cm} @

  default :
    @ \tikz[remember picture, overlay]\node[rounded corners, draw, xshift=-0.1cm, inner sep=5pt, anchor=west] {Kódrészlet \#2}; \vspace*{0.5cm} @ 
    return @ \tikz[remember picture, overlay]\node[rounded corners, draw, xshift=-0.1cm, inner sep=5pt, anchor=west, yshift=0.1cm] {Kifejezés \#2}; \vspace*{0.3cm} @
}
\end{lstlisting}
\end{minipage} 
\begin{minipage}[t]{0.6\textwidth}
\begin{lstlisting}[language=Java, numbers=none, breaklines=true]
Bounce<T> method1(GeneratorState<T> k) { @\vspace{0.2cm}@
  switch (@ \tikz[remember picture, overlay]\node[rounded corners, draw, xshift=-0.1cm, inner sep=5pt, anchor=west, yshift=0.1cm] {Szelektor}; \vspace*{0.3cm} \hspace*{1.35cm}@) {
    case @ \tikz[remember picture, overlay]\node[rounded corners, draw, xshift=-0.1cm, inner sep=5pt, anchor=west, yshift=0.1cm] {Konstans}; \vspace*{0.1cm} \hspace*{1.35cm}@:
      return cont(() -> method2(k));

    default :
      return cont(() -> method3(k));
  }
}

Bounce<T> method2(GeneratorState<T> k) {
  @ \tikz[remember picture, overlay]\node[rounded corners, draw, xshift=-0.1cm, inner sep=5pt, anchor=west] {Kódrészlet \#1}; \vspace*{0.3cm} @
  return cont(() -> method4(k), @ \tikz[remember picture, overlay]\node[rounded corners, draw, xshift=-0.1cm, inner sep=5pt, anchor=west, yshift=0.1cm] {Kifejezés \#1}; \vspace*{0.2cm} \hspace*{1.75cm} @);
}

Bounce<T> method3(GeneratorState<T> k) {
  @ \tikz[remember picture, overlay]\node[rounded corners, draw, xshift=-0.1cm, inner sep=5pt, anchor=west] {Kódrészlet \#2}; \vspace*{0.3cm} @
  return cont(() -> method4(k), @ \tikz[remember picture, overlay]\node[rounded corners, draw, xshift=-0.1cm, inner sep=5pt, anchor=west, yshift=0.1cm] {Kifejezés \#2}; \vspace*{0.2cm} \hspace*{1.75cm} @);
}

Bounce<T> method4(GeneratorState<T> k) {
  return cont(() -> k.apply(null));
}
\end{lstlisting} 
\end{minipage}
\end{mdframed}
\captionof{lstlisting}{\texttt{switch} utasítás transzformálása}
\end{center}
