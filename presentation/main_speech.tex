\documentclass[12pt, a4paper]{article}

\usepackage{t1enc}
\usepackage[utf8]{inputenc}
\usepackage[magyar]{babel}
\usepackage{amsmath}
\usepackage{listings}
\usepackage{color}
\usepackage{indentfirst}
\usepackage{xcolor}
\usepackage{colortbl}
\usepackage{caption}
\usepackage{ellipsis}
\usepackage{multirow}

\usepackage{url}
\usepackage{textcomp}

\usepackage{tikz}
\usetikzlibrary{shapes}
\usetikzlibrary{positioning}

\usepackage[linewidth=0.33pt, rightline=false, leftline=false, framemethod=tikz]{mdframed}

\begin{document}
\setcounter{section}{-1}
\section{Címlap}
Üdvözlöm a bíráló bizottság tagjait, a kar oktatóit és hallgatóit, valamint a megjelent érdeklődőket. Bagossy Attila vagyok, a dolgozatom és az előadásom címe \textit{Generátorok előállítása CPS-transzformációval Java nyelven}. A témavezetőim Dr. Battyányi Gyula Péter és Balla Tibor.

\section{Áttekintés}
Az előadásom összeállításakor úgy gondoltam, hogy nem fogom szigorúan követni a dolgozatom felépítését. A generátorokkal rokon szerkezetek, a \textit{continuation passing style}, valamint a kidolgozott eljárás mögötti megfontolások részletes ismertetése helyett inkább a gyakorlati oldalt szeretném megmutatni. 

Természetesen ahhoz, hogy az előadás önmagában is értelmezhető legyen, szükséges az alapfogalmak bevezetése. Ezt követően azonban a dolgozattól eltérve a transzformációs eljárás pontos bemutatása helyett érvelni szeretnék annak léte, s újdonságtartalma mellett. Először a generátorok más nyelvekben való elterjedtségére és széleskörű felhasználási lehetőségeire rámutatva fogom indokolni a \textit{Java}ban való bevezetésük szükségességét. Mivel nem az én eljárásom az első, mely ezt kívánja megvalósítani, a meglevő \textit{Java} nyelvű implementációk ismertetésével folytatom, kiemelve az eltéréseket. Az előadás második felében egy összetett kódrészlet transzformálásának lépésein kalauzolom végig a hallgatóságot, mely bemutatja az eljárást, s egyben rávilágít arra, hogy a \textit{CPS} mennyire elegáns megoldást ad egy ilyen komplex problémára.

\section{A generátor és rokonai}

Az áttekintést követően vegyük szemügyre, hogy mi az a \textit{generátor}, és milyen más eszközökkel van kapcsolatban.

\section{Szubrutin (\textit{subroutine})}

\begin{enumerate}
\item
A szubrutin az egyik leggyakrabban használt eszköz a programok szervezésére. Emiatt bemutatása szükségtelennek tűnhet, azonban ez az első lépcsőfok a generátorokhoz vezető úton. A szubrutinok alkalmasak gyakran használt kódrészletek kiemelésére, ami csökkenti a kódduplikációt. Emellett az implementáció elrejtésével növelhetik az absztrakciós szintet, és akár könyvtárak képezhetőek belőlük. Jelen esetben azonban a programvezérlésre kifejtett hatásukat szeretném kihangsúlyozni. Amikor meghívunk egy szubrutint, a vezérlés végighalad annak első utasításától az utolsóig, majd visszatér a hívóhoz. A hívó kód végrehajtása csak ezután folytatódhat tovább.
\item
Az ábra két oldala a hívót és a meghívott szubrutint ábrázolja. A nyilak a vezérlés irányát jelzik. Bár az ábrán nem látszik, de ki kell emelni, hogy a szubrutin végrehajtása minden egyes meghívásakor a legelső utasításánál kezdődik.
\end{enumerate}

\section{Korutin (\textit{coroutine})}

\begin{enumerate}
\item
A szubrutin mindig alárendelt szerepet játszott a hívó kódhoz képest. Ezzel szemben a korutinok alárendeltségi viszony mellett mellérendeltség kialakítására is alkalmasak, akár meghatározhatatlanná téve, hogy valójában ki hív kit. Ezt az teszi lehetővé, hogy futásuk felfüggeszthető és folytatható, és a hívások között megőrzik a lokális változóik értékét. Attól függően, hogy egy korutin tetszőleges másik korutint hívhat meg, vagy pedig muszáj visszaadnia valamikor a vezérlést az őt hívó korutinnak, beszélhetünk szimmetrikus és aszimmetrikus korutinokról.
\item
A fólián látható kódrészletekben szimmetrikus korutinok szerepelnek, melyek a \texttt{yield} utasítás segítségével adják át egymásnak a vezérlést. Miután \texttt{A} meghívta \texttt{B}-t, felfüggesztésre kerül, mindaddig, míg \texttt{B} vagy egy másik korutin újra meg nem hívja. Ekkor a szubrutinoktól eltérően ugyanonnan fog folytatódni a végrehajtás, ahol abbamaradt, tehát a \texttt{B}-t hívó \texttt{yield}-től.
\end{enumerate}

\section{Generátor (\textit{generator})}

A korutinok ismeretében már kifejezhető a generátor is, mely egy olyan aszimmetrikus korutin, mely egy sorozatot állít elő. A legnagyszerűbb tulajdonsága, hogy egyszerre csak egy elemet generál, ami azt jelenti, hogy lustán képzi a sorozatot. Ennek köszönhetően akár végtelen sorozatok reprezentálására alkalmas eszközt kínál. Megjegyzendő, hogy néhány programozási nyelv iterátornak nevezi a generátorokat.

Az ábra jobboldalán az ábécét előállító generátor szerepel, a baloldalán pedig ennek a meghívása egy \texttt{for} ciklus segítségével. Minden egyes iteráció alkalmával egy újabb elemet kérünk a generátortól, mindaddig, amíg a generált sorozat véget nem ér.

\section{Generátorok ismert nyelvekben}

Miután megismertük, mi az a generátor, folytassuk a támogatás vizsgálatával. Mely nyelvekben jelenik meg, milyen szerepet játszik? Hogyan könnyíti meg a programozók munkáját?

\section{Támogatást biztosító nyelvek}

A ma is ismert generátorokhoz hasonló szerkezetek először az \textit{IPL-V} és az \textit{Alphard} nyelvekben jelentek meg. Jelenleg használatos formájukat pedig a Barbara Liskov és munkatársai által kifejlesztett \textit{CLU} nyelvben nyerték el, bár a nyelv iterátornak nevezi őket. A \textit{CLU}-ban jelent meg először \textit{yield} kulcsszó használata.

A TIOBE minden hónapban kiadott, programozási nyelvek népszerűségét mérő listájának 10 legnépszerűbb nyelve között 5 olyan is van, mely nyelvi szintű támogatást ad a generátorok írásához. Ezek a C\#, a Python, a VB.NET, a PHP és a JavaScript. Ez mindenképpen figyelemre méltó, hiszen ezek közül például a C\# a Java közvetlen konkurensének tekinthető.

\section{Felhasználási lehetőségek -- 1}

A generátor egy elegáns szerkezet lehet a programozók eszköztárában. Nem csak elegáns azonban, hanem rendkívül erőteljes is, számtalan felhasználási lehetőséggel.

A sorozat fogalmat itt most nem matematikai értelemben használjuk, hanem egyszerűen csak egymást követő, azonos típusú elemeket jelent. Alkalmas a generátor végtelen sorozatok előállítására, ilyen például a Fibonacci-sorozat, vagy az összes prímszám. Használható véletlen értékek forrásaként. Ilyen formában alkalmazza a generátorokat a QuickCheck tesztelési keretrendszer is.

Véges sorozatok képzésére gyakran ott is bevethető a generátor, ahol nem is sejtenénk. Ilyen lehet például a reguláris kifejezések illesztése. A generátor mindig a következő illeszkedő karaktersorozatot adja vissza, lustán képezve valójában az összes illeszkedést. Fájlok beolvasására is alkalmazhatóak a generátorok, mindig csak egy sort vagy darabot beolvasva, \textit{on demand} módon, azaz igény szerint. Egy teljesen különböző ötlet paraméteres görbék pontjainak előállítása. A kliens kódnak semmit sem kell tudnia a görbe paramétertartományáról és megvalósítási részleteiről, csak sorban kéri a pontokat, amiket a generátor egyenként kiszámol.

\section{Felhasználási lehetőségek -- 2}

Az előző példák a generátorok hagyományos felhasználási lehetőségeit szemléltették. Azonban a generátorok aszimmetrikus korutin voltát még sokféleképpen ki lehet aknázni.

Meglepő lehet, de szimmetrikus korutinok megvalósítására is képesek. Ehhez egy \textit{trampoline} szükséges, melyen keresztül a generátorok hívni tudják egymást. Az \texttt{async/await} szerkezet leegyszerűsíti az aszinkron kód írását. A \textit{Babel} \textit{JavaScript} \textit{transpiler} valójában generátor-alapú kódot hoz létre az \texttt{async/await} szimulálására. Konkurens végrehajtásra is lehetőséget adnak a generátorok, mikroszálak segítségével. Ezek az operációs rendszer szintű szálaknál sokkal alacsonyabb erőforrásigénnyel rendelkeznek, és kooperatív multitaszkingot valósítanak meg.

\section{Korábbi \textit{Java} implementációk}

A generátorok \textit{Java}ban történő megvalósítására már több próbálkozás is történt. Mi ezekkel a probléma, miért érdemes még egy implementációt készíteni?

\section{Általános problémák}
\begin{enumerate}
    \item
    A rendelkezésre álló könyvtárak többségét már nem támogatják, a legutolsó módosítás jellemzően még a \textit{Java SE} 6-os vagy 7-es verziója idején történt. Ennek folytán bizonyos szerkezetek támogatása, mint például a \textit{try-with-resources}, a lambda függvények, vagy az interfészek \textit{default} metódusai nincsenek jelen ezen könyvtárakban.
    \item
    Többségében kényelmetlen, akár átgondolatlan interfészt kell a programozónak használnia. Természetesen a \textit{yield} kulcsszót csak a \textit{Java} fordító módosításával lehet bevezetni, amire egyik könyvtár sem vállalkozik. Más megoldáshoz folyamodva, legtöbbször különféle függvényhívásokkal érhető el a generátorokra jellemző működés.
    \item
    A különböző megvalósítások mögött sokféle, eltérő technika áll. Van példa bájtkód-instrumentációra, szálak használatára, illetve egyszerű programkönyvtárakra. Ezek egyike sem mondható optimálisnak. A bájtkód-instrumentáció kevésbé megbízható, mint az annotáció-feldolgozás, mivel a fordító nem ellenőrzi a generált kódot. Érvénytelen bájtkód esetén a \textit{JVM} megtagadhatja az osztály betöltését. A szálak elsősorban erőforrás-igényüket tekintve okoznak gondot. Az egyszerű programkönyvtárak egy ősosztályt biztosítanak, melynek gyermekei lesznek a generátorok. Ez nem kifejezetten elegáns és rugalmas megoldás.
\end{enumerate}

\section{jyield}
A kódrészlet a \textit{jyield} könyvtár példakódjainak egyike. Ez a könyvtár 7 éve lett frissítve utoljára, még a \textit{Java SE} 6 idején. Bájtkód-instrumentációt használ, és a \textit{Yield} osztály statikus metódusait kell hívni, ahogy a 8. és a 10. sorokban látható. A metódusok használata szemantikailag hibás, hiszen míg a \textit{yield} egy utasítás, addig a metódusok kifejezések. Ez félreértéseket, hibás működést okozhat.

\section{java-generator-functions}
A \textit{java-generator-functions} könyvtár egy ősosztályt biztosít, melynek gyermekosztályai lehetnek a generátorok. Ez kisebbfajta kényelmetlenséget jelent, hiszen nem elég csak egy annotációt elhelyezni a megfelelő metóduson. A könyvtár a háttérben szálakat használ, amiknek a száma korlátozott, és memóriaigényük sem elhanyagolható.

\section{lombok-pg}
A \textit{lombok-pg} az elterjedt \textit{Lombok} könyvtár kiegészítéseit tartalmazza. Azonban a fejlesztése mintegy 5 éve megszakadt, és a jelenlegi \textit{Lombok} verzióval már nem kompatibilis. Itt is a megjelenik a szemantikailag megkérdőjelezhető függvény használata. Emellett a generátor paramétereit kötelező a \textit{final} módosítóval ellátni. Előnye viszont ennek a megoldásnak, hogy annotáció-feldolgozást használ, és egy állapotgépnek megfelelő kódot generál. Ez tekinthető az egyik legjobb megoldásnak.

\section{Az eljárás egy példán keresztül}
Az előadás következő részében a dolgozatom eredményét jelentő eljárást fogom bemutatni egy példán keresztül.

\section{Legfontosabb jellemzők}

A példa előtt címszavakban vázolnám az eljárás legfontosabb jellemzőit.

\begin{enumerate}
    \item
    A módszer a \textit{Pluggable Annotation Processing API}-t használja, mely a fordítási folyamatba ékelve teszi lehetővé az annotációkon alapuló kódanalízist, és forráskódgenerálást. Az implementáció része egy annotáció, melyet azokon a metódusokon kell elhelyezni, melyeket generátorrá szeretnénk alakítani. Ezután a könyvtár a fordítás közben megkeresi ezeket a metódusokat, és elvégzi a szükséges transzformációkat.
    \item
    Különböző függvényhívások helyett a könyvtárat használó programozó a jól ismert \texttt{return} utasítással adhat vissza értékeket a generátorokból. E módon a \texttt{yield} szemantikájával ruházzuk fel a \texttt{return}-t, ezeknél fog megállni a generátor, majd innen fog tovább folytatódni.
    \item
    A \textit{CPS} elsősorban a funkcionális programozási nyelvekben elterjedt stílus, de használják fordítóprogramok belső reprezentációjaként is. Előnye, hogy alkalmazásakor a programvezérlés teljesen explicitté válik, könnyen manipulálható a futás iránya. Egy ilyen könyvtár esetén pont erre van szükségünk. Fordítási időben folyamatosan követni fogjuk a \textit{continuation}öket, futási időben pedig egy \textit{trampoline} segítségével valósítjuk meg a technikát. Az említett \textit{continuation} nem más, mint a hátralevő teendőket jelképező struktúra.
\end{enumerate}

\section{Prímek generálása}

\begin{enumerate}
    \item
    A kódrészlet prímszámok végtelen sorozatát generálja kettőtől kezdve. A megvalósítás egyetlen célja, hogy szemléltesse az eljárás működését, ebből fakad viszonylag primitív volta.
    \item
    A generátor először kettőt fog visszadni, az ötödik sorban található \texttt{return} utasításnak köszönhetően.
    \item
    Ezt követi úgymond az összes többi prímszám a 20. sor \texttt{return}je miatt.  Az előállított prímek tárolásra kerülnek, egy páratlan szám akkor lesz prím, ha az összes eltárolt érték egyike sem osztja.
\end{enumerate}

\section{Prímek generálása -- Vágási pontok}

\begin{enumerate}
   \item
   A generátorok jellemzője, hogy a \texttt{yield} segítségével futásuk felfüggeszthető, majd ettől a ponttól újra folytatható. Ennek megvalósításához szét kell vágnunk az eredeti metódust kisebb darabokra. A futás folytatásakor mindig a következő darabot fogjuk elővenni. Hol kell azonban elvégezni ezeket a vágásokat? Úgynevezett explicit és implicit vágási pontokat azonosíthatunk a kódban.
   \item
   Explicit vágási pont lesz a \texttt{yield}, azaz valójában \texttt{return} utasítás. Ezt látva biztosan vágnunk kell, mivel a \texttt{yield} függeszti fel a generátort és ad vissza egy értéket.
   \item
   Implicit vágási pontokat a különböző vezérlési szerkezetek vezethetnek be. Fontos megjegyezni, hogy ezeket a vágási pontokat csak akkor kell figyelembe venni, ha az adott vezérlési szerkezet tartalmaz egy explicit vágási pontot is. Az egyetlen vezérlési struktúra, mely \texttt{return}t tartalmaz, a \textit{do-while} ciklus, így csak ennek az eleje és a vége lesz implicit vágási pont.
   \item
   A \textit{do-while} ciklust tehát darabokra kell bontanunk. Ez a tartalmazott más vezérlési szerkezeteket alapesetben nem érinti, azonban a 14. sorban található, \texttt{loop} címkével ellátott \texttt{continue} utasítás a szétdarabolt ciklust szeretné léptetni.
   \item
   Emiatt szükséges a \texttt{continue}-t befoglaló elágazás és \texttt{foreach} ciklus feldarabolása is. Az így létrejövő kis metódustöredékek lesznek a \textit{continuation}ök, melyekkel mindig folytatódni fog a futás.
\end{enumerate}

\section{}

\end{document}
