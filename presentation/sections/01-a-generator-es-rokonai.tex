\section{A generátor és rokonai}


\begin{frame}[fragile]{Szubrutin \textit{(subroutine)}}
Alkalmas kódrészletek kiemelésére, csökkentve a duplikációt.

Meghívása után a hívó kód végrehajtása felfüggesztésre kerül a visszatérésig.
\\
\begin{center}
\begin{minipage}{.40\textwidth}
\begin{lstlisting}[title=Hívó, frame=t, escapechar=!]
!\tikz[remember picture] \node [] (a) {};!/*
  * Kódrészlet
  */
!\tikz[remember picture] \node [] (b) {};! add(2, 2); !\tikz[remember picture] \node [] (c) {};!
 /*
  * Kódrészlet
  */
!\tikz[remember picture] \node [] (f) {};!
\end{lstlisting}
\end{minipage}\hfill
\begin{minipage}{.50\textwidth}
\begin{lstlisting}[title=Szubrutin, frame=t, escapechar=!, showlines=true]
!\tikz[remember picture] \node [] (d) {};! int add(int a, int b)
  {
    /*
     * Kódrészlet
     */
!\tikz[remember picture] \node [] (e) {};! }


\end{lstlisting}
\end{minipage}
\begin{tikzpicture}[remember picture, overlay,
    every edge/.append style = { ->, thick, transparent, >=stealth, line width = 1pt }] 
  \draw (a.north) + (0, 0) coordinate(x1) edge (x1|-b.north);
  \draw (c.east) + (0, 0.15) edge (d.west);
  \draw (d.south) + (0, 0) edge (e.north);
  \draw (e.west) + (0, 0) edge (c.east) + (0, -0.15);
  \draw (b.south) + (0, 0) edge (f.north);
\end{tikzpicture} 
\end{center}
\par
\end{frame}


\begin{frame}[fragile]{Szubrutin \textit{(subroutine)}}
Alkalmas kódrészletek kiemelésére, csökkentve a duplikációt.

Meghívása után a hívó kód végrehajtása felfüggesztésre kerül a visszatérésig.
\\
\begin{center}
\begin{minipage}{.40\textwidth}
\begin{lstlisting}[title=Hívó, frame=t, escapechar=!]
!\tikz[remember picture] \node [] (a) {};!/*
  * Kódrészlet
  */
!\tikz[remember picture] \node [] (b) {};! add(2, 2); !\tikz[remember picture] \node [] (c) {};!
 /*
  * Kódrészlet
  */
!\tikz[remember picture] \node [] (f) {};!
\end{lstlisting}
\end{minipage}\hfill
\begin{minipage}{.50\textwidth}
\begin{lstlisting}[title=Szubrutin, frame=t, escapechar=!, showlines=true]
!\tikz[remember picture] \node [] (d) {};! int add(int a, int b)
  {
    /*
     * Kódrészlet
     */
!\tikz[remember picture] \node [] (e) {};! }


\end{lstlisting}
\end{minipage}
\begin{tikzpicture}[remember picture, overlay,
    every edge/.append style = { ->, thick, >=stealth, line width = 1pt }] 
  \draw (a.north) + (0, 0) coordinate(x1) edge (x1|-b.north);
  \draw (c.east) + (0, 0.15) edge (d.west);
  \draw (d.south) + (0, 0) edge (e.north);
  \draw (e.west) + (0, 0) edge (c.east) + (0, -0.15);
  \draw (b.south) + (0, 0) edge (f.north);
\end{tikzpicture} 
\end{center}
\par
\end{frame}


\begin{frame}[fragile]{Korutin (\textit{coroutine})}
A végrehajtás mindig ott folytatódik, ahol a legutóbbi hívás esetén abbamaradt.

A lokális változók értéke megőrződik a hívások között.
\\
\begin{center}
\begin{minipage}{.40\textwidth}
\begin{lstlisting}[escapechar=!, keywords={coroutine, yield}]
coroutine A()
{
    /* ... */
    yield B; !\tikz[remember picture] \node [] (a) {};!
    /* ... */!\tikz[remember picture] \node [] (d) {};!
    yield B; !\tikz[remember picture] \node [] (e) {};!
}
\end{lstlisting}
\end{minipage}\hfill
\begin{minipage}{.50\textwidth}
\begin{lstlisting}[escapechar=!, showlines=true, keywords={coroutine, yield}]
coroutine B()
{
    !\tikz[remember picture] \node [] (b) {};! /* ... */
    !\tikz[remember picture] \node [] (c) {};! yield A; 
    !\tikz[remember picture] \node [] (f) {};! /* ... */
}

\end{lstlisting}
\end{minipage}
\begin{tikzpicture}[remember picture, overlay,
    every edge/.append style = { ->, thick, transparent, >=stealth, line width = 1pt }] 
  \draw (a.east) + (0, 0) edge (b.west);
  \draw (c.west) + (0, 0) edge (d.east);
  \draw (e.east) + (0, 0) edge (f.west);
\end{tikzpicture} 
\end{center}
\par
\end{frame}


\begin{frame}[fragile]{Korutin (\textit{coroutine})}
A végrehajtás mindig ott folytatódik, ahol a legutóbbi hívás esetén abbamaradt.

A lokális változók értéke megőrződik a hívások között.
\\
\begin{center}
\begin{minipage}{.40\textwidth}
\begin{lstlisting}[escapechar=!, keywords={coroutine, yield}]
coroutine A()
{
    /* ... */
    yield B; !\tikz[remember picture] \node [] (a) {};!
    /* ... */!\tikz[remember picture] \node [] (d) {};!
    yield B; !\tikz[remember picture] \node [] (e) {};!
}
\end{lstlisting}
\end{minipage}\hfill
\begin{minipage}{.50\textwidth}
\begin{lstlisting}[escapechar=!, showlines=true, keywords={coroutine, yield}]
coroutine B()
{
    !\tikz[remember picture] \node [] (b) {};! /* ... */
    !\tikz[remember picture] \node [] (c) {};! yield A; 
    !\tikz[remember picture] \node [] (f) {};! /* ... */
}

\end{lstlisting}
\end{minipage}
\begin{tikzpicture}[remember picture, overlay,
    every edge/.append style = { ->, thick, >=stealth, line width = 1pt }] 
  \draw[orange] (a.east) + (0, 0) edge (b.west);
  \draw[transparent] (c.west) + (0, 0) edge (d.east);
  \draw[transparent] (e.east) + (0, 0) edge (f.west);
\end{tikzpicture} 
\end{center}
\par
\end{frame}


\begin{frame}[fragile]{Korutin (\textit{coroutine})}
A végrehajtás mindig ott folytatódik, ahol a legutóbbi hívás esetén abbamaradt.

A lokális változók értéke megőrződik a hívások között.
\\
\begin{center}
\begin{minipage}{.40\textwidth}
\begin{lstlisting}[escapechar=!, keywords={coroutine, yield}]
coroutine A()
{
    /* ... */
    yield B; !\tikz[remember picture] \node [] (a) {};!
    /* ... */!\tikz[remember picture] \node [] (d) {};!
    yield B; !\tikz[remember picture] \node [] (e) {};!
}
\end{lstlisting}
\end{minipage}\hfill
\begin{minipage}{.50\textwidth}
\begin{lstlisting}[escapechar=!, showlines=true, keywords={coroutine, yield}]
coroutine B()
{
    !\tikz[remember picture] \node [] (b) {};! /* ... */
    !\tikz[remember picture] \node [] (c) {};! yield A; 
    !\tikz[remember picture] \node [] (f) {};! /* ... */
}

\end{lstlisting}
\end{minipage}
\begin{tikzpicture}[remember picture, overlay,
    every edge/.append style = { ->, thick, , >=stealth, line width = 1pt }] 
  \draw (a.east) + (0, 0) edge (b.west);
  \draw[orange] (c.west) + (0, 0) edge (d.east);
  \draw[transparent] (e.east) + (0, 0) edge (f.west);
\end{tikzpicture} 
\end{center}
\par
\end{frame}


\begin{frame}[fragile]{Korutin (\textit{coroutine})}
A végrehajtás mindig ott folytatódik, ahol a legutóbbi hívás esetén abbamaradt.

A lokális változók értéke megőrződik a hívások között.
\\
\begin{center}
\begin{minipage}{.40\textwidth}
\begin{lstlisting}[escapechar=!, keywords={coroutine, yield}]
coroutine A()
{
    /* ... */
    yield B; !\tikz[remember picture] \node [] (a) {};!
    /* ... */!\tikz[remember picture] \node [] (d) {};!
    yield B; !\tikz[remember picture] \node [] (e) {};!
}
\end{lstlisting}
\end{minipage}\hfill
\begin{minipage}{.50\textwidth}
\begin{lstlisting}[escapechar=!, showlines=true, keywords={coroutine, yield}]
coroutine B()
{
    !\tikz[remember picture] \node [] (b) {};! /* ... */
    !\tikz[remember picture] \node [] (c) {};! yield A; 
    !\tikz[remember picture] \node [] (f) {};! /* ... */
}

\end{lstlisting}
\end{minipage}
\begin{tikzpicture}[remember picture, overlay,
    every edge/.append style = { ->, thick, >=stealth, line width = 1pt }] 
  \draw (a.east) + (0, 0) edge (b.west);
  \draw (c.west) + (0, 0) edge (d.east);
  \draw[orange] (e.east) + (0, 0) edge (f.west);
\end{tikzpicture} 
\end{center}
\par
\end{frame}


\begin{frame}[fragile]{Generátor (\textit{generator})}
Aszimmetrikus korutin, amely elemek sorozatát állítja elő.

\hfill \\

Minden egyes hívás alkalmával a sorozat egy elemét képzi.

\begin{center}
\begin{minipage}{.40\textwidth}
\begin{lstlisting}[escapechar=!, showlines=true, keywords={for, in, print}]
for ch in alphabet()
{
    print ch
}



\end{lstlisting}
\end{minipage}\hfill
\begin{minipage}{.50\textwidth}
\begin{lstlisting}[escapechar=!, showlines=true, keywords={generator, yield}]
generator alphabet()
{
    yield 'a';
    yield 'b';
    yield 'c';
    /* ... */
}
\end{lstlisting}
\end{minipage}
\end{center}
\par
\end{frame}