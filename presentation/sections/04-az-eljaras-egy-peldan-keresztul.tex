\section{Az eljárás egy példán keresztül}

\begin{frame}{Az eljárás jellemzői}
A \textit{Pluggable Annotation Processing} API-t használja


\hfill \\

\pause
A \texttt{return} utasítás jelképezi a \texttt{yield} utasítást


\hfill \\

\pause
\textit{Continuation Passing Style}-alapú megvalósítás
\end{frame}


\begin{frame}[fragile]{Prímek generálása}
\begin{center}
\begin{lstlisting}[language=java, xleftmargin=15pt,
        basicstyle=\fontsize{7}{9}\selectfont,
        numbers=left,
        numbersep=5pt, escapechar=!,
        linebackgroundcolor={
            \btLstHL<1>{30} % No highlighting
            \btLstHL<2>{5}
            \btLstHL<3>{20}
        }]
@Generator
private Stream<Integer> generatePrimes() {
    LinkedList<Integer> primes = new LinkedList<>();
    primes.add(2);
    return 2;

    int current = 1;

    loop:
    do {
        current += 2;
        for (int i : primes) {
            if (current % i == 0) {
                continue loop;
            }
        }

        primes.add(current);

        return current;
    } while (true);
}
\end{lstlisting}
\end{center}
\par
\end{frame}


\begin{frame}[fragile]{Prímek generálása -- Vágási pontok}
\begin{center}
\begin{lstlisting}[language=java, xleftmargin=15pt,
        basicstyle=\fontsize{7}{9}\selectfont,
        numbers=left,
        numbersep=5pt, escapechar=!,
        linebackgroundcolor={
            \btLstHL<1>{30} % No highlighting
            \btLstHL<2>{5, 20}
            \btLstHL<3>{10, 21}
            \btLstHL<4>{14}
            \btLstHL<5>{12-16}
        }]
@Generator
private Stream<Integer> generatePrimes() {
    LinkedList<Integer> primes = new LinkedList<>();
    primes.add(2);
    return 2;

    int current = 1;

    loop:
    do {
        current += 2;
        for (int i : primes) {
            if (current % i == 0) {
                continue loop;
            }
        }

        primes.add(current);

        return current;
    } while (true);
}
\end{lstlisting}
\end{center}
\par
\end{frame}


\begin{frame}{A vezérlés útja}
\begin{center}
\begin{tikzpicture}
\begin{scope}
    \draw[thick, rounded corners=6pt]  (-4.55, -1.57) rectangle (4.5, -6.67);

    \draw[thick, rounded corners=6pt]  (-4.3, -3.10) rectangle (4.19, -4.4);
\end{scope}

\begin{scope}
  \node at (-3.75, -1.4) {\texttt{do-while}};

  \node at (-3.6, -2.93) {\texttt{foreach}};
\end{scope}

\begin{scope}[every node/.style={thick,draw, rounded corners=4pt}]
  \node (1) at (0, 0) {\texttt{primes\_1}};
  \node (2) at (0, -1) {\texttt{primes\_2}};
  \node (3) at (0, -2.5) {\texttt{primes\_3}};
  \node (5) at (0, -5) {\texttt{primes\_5}};
  \node (4) at (0, -6) {\texttt{primes\_4}};
  \node (6) at (-3, -3.75) {\texttt{primes\_6}};
  \node (7) at (0, -3.75) {\texttt{primes\_7}};
  \node (8) at (3, -3.75) {\texttt{primes\_8}};
  \node (0) at (0, -7.33) {\texttt{primes\_0}};
\end{scope}

\begin{scope}[>={Stealth[black]},
              every edge/.style={draw=orange, very thick}]
    \path [->] (1) edge (2);
    \path [->] (2) edge (3);
    \path [->] (3) edge (6);
    \path [->] (6) edge (7);
    \path [->] (7) edge[bend right=30] (8);
    \path [->] (8) edge[bend right=30] (7);
    \path [->] (7) edge (5);
    \path [->] (5) edge (4);
    \path [->] (4) edge (0);
    \path (4) edge[bend right=60]  (4.33, -4);
    \path[->] (4.33, -4) edge[bend right=70] (3);
\end{scope}
\end{tikzpicture}
\end{center}
\end{frame}


\lstset{language=java, xleftmargin=15pt,
        basicstyle=\scriptsize,
        numbers=left,
        numbersep=5pt, escapechar=!}

\begin{frame}[fragile]{A transzformáció lépései -- 1}
\begin{lstlisting}
private Bounce<Integer> primes_0(GenState<Integer> k) {
    return Bounce.cont(()->k.apply(GenState.empty()));
}
\end{lstlisting}
\end{frame}


\begin{frame}[fragile]{A transzformáció lépései -- 2}
\begin{center}\textbf{Eredeti kód}\end{center}
\begin{lstlisting}[linebackgroundcolor={
            \btLstHL<1>{30} % No highlighting
            \btLstHL<2>{1}
            \btLstHL<3>{3}
        }]
 LinkedList<Integer> primes = new LinkedList<>();
 primes.add(2);
 return 2;
\end{lstlisting}

\begin{center}\textbf{Transzformált kód}\end{center}
\begin{lstlisting}[linebackgroundcolor={
            \btLstHL<1>{30} % No highlighting
            \btLstHL<2>{2}
            \btLstHL<3>{4}
        }]
 private Bounce<Integer> primes_1(GenState<Integer> k) {
     primes = new LinkedList<>();
     primes.add(2);
     return Bounce.cont(()->primes_2(k), 2);
 }
\end{lstlisting}
\end{frame}


\begin{frame}[fragile]{A transzformáció lépései -- 3}
\begin{center}\textbf{Eredeti kód}\end{center}
\begin{lstlisting}[linebackgroundcolor={
            \btLstHL<1>{30} % No highlighting
            \btLstHL<2>{6}
            \btLstHL<3>{30} % No highlighting
            \btLstHL<4>{30} % No highlighting
        }]
 loop:
 do {
     /*
      * Törzs
      */
 } while (true);
\end{lstlisting}

\begin{center}\textbf{Transzformált kód}\end{center}
\begin{lstlisting}[linebackgroundcolor={
            \btLstHL<1>{30} % No highlighting
            \btLstHL<2>{2}
            \btLstHL<3>{3}
            \btLstHL<4>{5}
        }]
 private Bounce<Integer> primes_4(GenState<Integer> k) {
     if (true) {
         return Bounce.cont(()->primes_3(k));
     }
     return Bounce.cont(()->primes_0(k));
 }
\end{lstlisting}
\end{frame}


\begin{frame}[fragile]{A transzformáció lépései -- 4}
\begin{center}\textbf{Eredeti kód}\end{center}
\begin{lstlisting}[linebackgroundcolor={
            \btLstHL<1>{1}
        }]
 current += 2;
 for (int i : primes) {
     if (current % i == 0) {
         continue loop;
     }
 }
 
 primes.add(current);
 
 return current;
\end{lstlisting}

\begin{center}\textbf{Transzformált kód}\end{center}
\begin{lstlisting}
 private Bounce<Integer> primes_3(GenState<Integer> k) {
     current += 2;
     return Bounce.cont(()->primes_6(k));
 }
\end{lstlisting}
\end{frame}

\begin{frame}[fragile]{A transzformáció lépései -- 4}
\begin{center}\textbf{Eredeti kód}\end{center}
\begin{lstlisting}[linebackgroundcolor={
            \btLstHL<1>{8-10}
        }]
 current += 2;
 for (int i : primes) {
     if (current % i == 0) {
         continue loop;
     }
 }
 
 primes.add(current);
 
 return current;
\end{lstlisting}

\begin{center}\textbf{Transzformált kód}\end{center}
\begin{lstlisting}
 private Bounce<Integer> primes_5(GenState<Integer> k) {
     primes.add(current);
     return Bounce.cont(()->primes_4(k), current);
 }
\end{lstlisting}
\addtocounter{framenumber}{-1}
\end{frame}



\begin{frame}[fragile]{A transzformáció lépései -- 5}
\begin{center}\textbf{Eredeti kód}\end{center}
\begin{lstlisting}[linebackgroundcolor={
            \btLstHL<1>{1}
        }]
 for (int i : primes) {
     if (current % i == 0) {
         continue loop;
     }
 }
\end{lstlisting}

\begin{center}\textbf{Transzformált kód}\end{center}
\begin{lstlisting}
 private Bounce<Integer> primes_6(GeState<Integer> k) {
     iterator = CPSUtil.iterator(primes);
     return Bounce.cont(()->primes_7(k));
 }
\end{lstlisting}
\end{frame}


\begin{frame}[fragile]{A transzformáció lépései -- 5}
\begin{center}\textbf{Eredeti kód}\end{center}
\begin{lstlisting}[linebackgroundcolor={
            \btLstHL<1>{1}
        }]
 for (int i : primes) {
     if (current % i == 0) {
         continue loop;
     }
 }
\end{lstlisting}

\begin{center}\textbf{Transzformált kód}\end{center}
\begin{lstlisting}
 private Bounce<Integer> primes_7(GenState<Integer> k) {
     if (iterator.hasNext()) {
         i = iterator.next();
         return Bounce.cont(()->primes_8(k));
     }
     return Bounce.cont(()->primes_5(k));
 }
\end{lstlisting}
\addtocounter{framenumber}{-1}
\end{frame}


\begin{frame}[fragile]{A transzformáció lépései -- 5}
\begin{center}\textbf{Eredeti kód}\end{center}
\begin{lstlisting}[linebackgroundcolor={
            \btLstHL<1>{2-4}
        }]
 for (int i : primes) {
     if (current % i == 0) {
         continue loop;
     }
 }
\end{lstlisting}

\begin{center}\textbf{Transzformált kód}\end{center}
\begin{lstlisting}
 private Bounce<Integer> primes_8(GenState<Integer> k) {
     if (current % i == 0) {
         return Bounce.cont(()->primes_4(k));
     }
     return Bounce.cont(()->primes_7(k));
 }
\end{lstlisting}
\addtocounter{framenumber}{-1}
\end{frame}        


\begin{frame}{A vezérlés útja}
\begin{center}
\begin{tikzpicture}
\begin{scope}
    \draw[thick, rounded corners=6pt]  (-4.55, -1.57) rectangle (4.5, -6.67);

    \draw[thick, rounded corners=6pt]  (-4.3, -3.10) rectangle (4.19, -4.4);
\end{scope}

\begin{scope}
  \node at (-3.75, -1.4) {\texttt{do-while}};

  \node at (-3.6, -2.93) {\texttt{foreach}};
\end{scope}

\begin{scope}[every node/.style={thick,draw, rounded corners=4pt}]
  \node (1) at (0, 0) {\texttt{primes\_1}};
  \node (2) at (0, -1) {\texttt{primes\_2}};
  \node (3) at (0, -2.5) {\texttt{primes\_3}};
  \node (5) at (0, -5) {\texttt{primes\_5}};
  \node (4) at (0, -6) {\texttt{primes\_4}};
  \node (6) at (-3, -3.75) {\texttt{primes\_6}};
  \node (7) at (0, -3.75) {\texttt{primes\_7}};
  \node (8) at (3, -3.75) {\texttt{primes\_8}};
  \node (0) at (0, -7.33) {\texttt{primes\_0}};
\end{scope}

\begin{scope}[>={Stealth[black]},
              every edge/.style={draw=orange, very thick}]
    \path [->] (1) edge (2);
    \path [->] (2) edge (3);
    \path [->] (3) edge (6);
    \path [->] (6) edge (7);
    \path [->] (7) edge[bend right=30] (8);
    \path [->] (8) edge[bend right=30] (7);
    \path [->] (7) edge (5);
    \path [->] (5) edge (4);
    \path [->] (4) edge (0);
    \path (4) edge[bend right=60]  (4.33, -4);
    \path[->] (4.33, -4) edge[bend right=70] (3);
\end{scope}
\end{tikzpicture}
\end{center}
\end{frame}
