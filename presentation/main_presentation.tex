\documentclass{beamer}

\usepackage{t1enc}
\usepackage[utf8]{inputenc}
\usepackage[magyar]{babel}
\usepackage[useregional]{datetime2}

\usepackage{tikz}
\usetikzlibrary{positioning}
\usetikzlibrary{shapes}
\usetikzlibrary{arrows.meta}
\usepackage{listings}
\usepackage{lstlinebgrd}
\usepackage{pgf, pgffor}
\usepackage{caption}

% From http://tex.stackexchange.com/a/85628

\makeatletter
%%%%%%%%%%%%%%%%%%%%%%%%%%%%%%%%%%%%%%%%%%%%%%%%%%%%%%%%%%%%%%%%%%%%%%%%%%%%%%
%
% \btIfInRange{number}{range list}{TRUE}{FALSE}
%
% Test in int number <number> is element of a (comma separated) list of ranges
% (such as: {1,3-5,7,10-12,14}) and processes <TRUE> or <FALSE> respectively

\newcount\bt@rangea
\newcount\bt@rangeb

\newcommand\btIfInRange[2]{%
    \global\let\bt@inrange\@secondoftwo%
    \edef\bt@rangelist{#2}%
    \foreach \range in \bt@rangelist {%
        \afterassignment\bt@getrangeb%
        \bt@rangea=0\range\relax%
        \pgfmathtruncatemacro\result{ ( #1 >= \bt@rangea) && (#1 <= \bt@rangeb) }%
        \ifnum\result=1\relax%
            \breakforeach%
            \global\let\bt@inrange\@firstoftwo%
        \fi%
    }%
    \bt@inrange%
}
\newcommand\bt@getrangeb{%
    \@ifnextchar\relax%
        {\bt@rangeb=\bt@rangea}%
        {\@getrangeb}%
}
\def\@getrangeb-#1\relax{%
    \ifx\relax#1\relax%
        \bt@rangeb=100000%   \maxdimen is too large for pgfmath
    \else%
        \bt@rangeb=#1\relax%
    \fi%
}

%%%%%%%%%%%%%%%%%%%%%%%%%%%%%%%%%%%%%%%%%%%%%%%%%%%%%%%%%%%%%%%%%%%%%%%%%%%%%%
%
% \btLstHL<overlay spec>{range list}
%
% TODO BUG: \btLstHL commands can not yet be accumulated if more than one overlay spec match.
% 
\newcommand<>{\btLstHL}[1]{%
  \only#2{\btIfInRange{\value{lstnumber}}{#1}{\color{orange!30}\def\lst@linebgrdcmd{\color@block}}{\def\lst@linebgrdcmd####1####2####3{}}}%
}%
\makeatother


\lstset{
  language=C,
  basicstyle=\small,
  breaklines=true
  }

\lstset{literate=
 {á}{{\'a}}1 {é}{{\'e}}1 {í}{{\'i}}1 {ó}{{\'o}}1 {ú}{{\'u}}1
 {Á}{{\'A}}1 {É}{{\'E}}1 {Í}{{\'I}}1 {Ó}{{\'O}}1 {Ú}{{\'U}}1
 {ö}{{\"o}}1 {ü}{{\"u}}1 {Ö}{{\"O}}1 {Ü}{{\"U}}1
 {ű}{{\H{u}}}1 {Ű}{{\H{U}}}1 {ő}{{\H{o}}}1 {Ő}{{\H{O}}}1
}

\usetheme{metropolis}
\title{Generátorok előállítása \textit{CPS}-transzformációval Java nyelven}
\date{2017. április 20.}
\author[My name]{Bagossy Attila\\ \footnotesize Témavezetők: Dr. Battyányi Péter, Balla Tibor \vspace{5em}}
\institute{Debreceni Egyetem, Informatikai Kar, Számítógéptudományi Tanszék}

\begin{document}
  \maketitle

  \begin{frame}{Áttekintés}
    \setbeamertemplate{section in toc}[sections numbered]
    \tableofcontents[hideallsubsections]
  \end{frame}

  \section{A generátor és rokonai}


\begin{frame}[fragile]{Szubrutin \textit{(subroutine)}}
Alkalmas kódrészletek kiemelésére, csökkentve a duplikációt.

Meghívása után a hívó kód végrehajtása felfüggesztésre kerül a visszatérésig.
\\
\begin{center}
\begin{minipage}{.40\textwidth}
\begin{lstlisting}[title=Hívó, frame=t, escapechar=!]
!\tikz[remember picture] \node [] (a) {};!/*
  * Kódrészlet
  */
!\tikz[remember picture] \node [] (b) {};! add(2, 2); !\tikz[remember picture] \node [] (c) {};!
 /*
  * Kódrészlet
  */
!\tikz[remember picture] \node [] (f) {};!
\end{lstlisting}
\end{minipage}\hfill
\begin{minipage}{.50\textwidth}
\begin{lstlisting}[title=Szubrutin, frame=t, escapechar=!, showlines=true]
!\tikz[remember picture] \node [] (d) {};! int add(int a, int b)
  {
    /*
     * Kódrészlet
     */
!\tikz[remember picture] \node [] (e) {};! }


\end{lstlisting}
\end{minipage}
\begin{tikzpicture}[remember picture, overlay,
    every edge/.append style = { ->, thick, transparent, >=stealth, line width = 1pt }] 
  \draw (a.north) + (0, 0) coordinate(x1) edge (x1|-b.north);
  \draw (c.east) + (0, 0.15) edge (d.west);
  \draw (d.south) + (0, 0) edge (e.north);
  \draw (e.west) + (0, 0) edge (c.east) + (0, -0.15);
  \draw (b.south) + (0, 0) edge (f.north);
\end{tikzpicture} 
\end{center}
\par
\end{frame}


\begin{frame}[fragile]{Szubrutin \textit{(subroutine)}}
Alkalmas kódrészletek kiemelésére, csökkentve a duplikációt.

Meghívása után a hívó kód végrehajtása felfüggesztésre kerül a visszatérésig.
\\
\begin{center}
\begin{minipage}{.40\textwidth}
\begin{lstlisting}[title=Hívó, frame=t, escapechar=!]
!\tikz[remember picture] \node [] (a) {};!/*
  * Kódrészlet
  */
!\tikz[remember picture] \node [] (b) {};! add(2, 2); !\tikz[remember picture] \node [] (c) {};!
 /*
  * Kódrészlet
  */
!\tikz[remember picture] \node [] (f) {};!
\end{lstlisting}
\end{minipage}\hfill
\begin{minipage}{.50\textwidth}
\begin{lstlisting}[title=Szubrutin, frame=t, escapechar=!, showlines=true]
!\tikz[remember picture] \node [] (d) {};! int add(int a, int b)
  {
    /*
     * Kódrészlet
     */
!\tikz[remember picture] \node [] (e) {};! }


\end{lstlisting}
\end{minipage}
\begin{tikzpicture}[remember picture, overlay,
    every edge/.append style = { ->, thick, >=stealth, line width = 1pt }] 
  \draw (a.north) + (0, 0) coordinate(x1) edge (x1|-b.north);
  \draw (c.east) + (0, 0.15) edge (d.west);
  \draw (d.south) + (0, 0) edge (e.north);
  \draw (e.west) + (0, 0) edge (c.east) + (0, -0.15);
  \draw (b.south) + (0, 0) edge (f.north);
\end{tikzpicture} 
\end{center}
\par
\addtocounter{framenumber}{-1}
\end{frame}


\begin{frame}[fragile]{Korutin (\textit{coroutine})}
A végrehajtás mindig ott folytatódik, ahol a legutóbbi hívás esetén abbamaradt.

A lokális változók értéke megőrződik a hívások között.
\\
\begin{center}
\begin{minipage}{.40\textwidth}
\begin{lstlisting}[escapechar=!, keywords={coroutine, yield}]
coroutine A()
{
    /* ... */
    yield B; !\tikz[remember picture] \node [] (a) {};!
    /* ... */!\tikz[remember picture] \node [] (d) {};!
    yield B; !\tikz[remember picture] \node [] (e) {};!
}
\end{lstlisting}
\end{minipage}\hfill
\begin{minipage}{.50\textwidth}
\begin{lstlisting}[escapechar=!, showlines=true, keywords={coroutine, yield}]
coroutine B()
{
    !\tikz[remember picture] \node [] (b) {};! /* ... */
    !\tikz[remember picture] \node [] (c) {};! yield A; 
    !\tikz[remember picture] \node [] (f) {};! /* ... */
}

\end{lstlisting}
\end{minipage}
\begin{tikzpicture}[remember picture, overlay,
    every edge/.append style = { ->, thick, transparent, >=stealth, line width = 1pt }] 
  \draw (a.east) + (0, 0) edge (b.west);
  \draw (c.west) + (0, 0) edge (d.east);
  \draw (e.east) + (0, 0) edge (f.west);
\end{tikzpicture} 
\end{center}
\par
\end{frame}


\begin{frame}[fragile]{Korutin (\textit{coroutine})}
A végrehajtás mindig ott folytatódik, ahol a legutóbbi hívás esetén abbamaradt.

A lokális változók értéke megőrződik a hívások között.
\\
\begin{center}
\begin{minipage}{.40\textwidth}
\begin{lstlisting}[escapechar=!, keywords={coroutine, yield}]
coroutine A()
{
    /* ... */
    yield B; !\tikz[remember picture] \node [] (a) {};!
    /* ... */!\tikz[remember picture] \node [] (d) {};!
    yield B; !\tikz[remember picture] \node [] (e) {};!
}
\end{lstlisting}
\end{minipage}\hfill
\begin{minipage}{.50\textwidth}
\begin{lstlisting}[escapechar=!, showlines=true, keywords={coroutine, yield}]
coroutine B()
{
    !\tikz[remember picture] \node [] (b) {};! /* ... */
    !\tikz[remember picture] \node [] (c) {};! yield A; 
    !\tikz[remember picture] \node [] (f) {};! /* ... */
}

\end{lstlisting}
\end{minipage}
\begin{tikzpicture}[remember picture, overlay,
    every edge/.append style = { ->, thick, >=stealth, line width = 1pt }] 
  \draw[orange] (a.east) + (0, 0) edge (b.west);
  \draw[transparent] (c.west) + (0, 0) edge (d.east);
  \draw[transparent] (e.east) + (0, 0) edge (f.west);
\end{tikzpicture} 
\end{center}
\par
\addtocounter{framenumber}{-1}
\end{frame}


\begin{frame}[fragile]{Korutin (\textit{coroutine})}
A végrehajtás mindig ott folytatódik, ahol a legutóbbi hívás esetén abbamaradt.

A lokális változók értéke megőrződik a hívások között.
\\
\begin{center}
\begin{minipage}{.40\textwidth}
\begin{lstlisting}[escapechar=!, keywords={coroutine, yield}]
coroutine A()
{
    /* ... */
    yield B; !\tikz[remember picture] \node [] (a) {};!
    /* ... */!\tikz[remember picture] \node [] (d) {};!
    yield B; !\tikz[remember picture] \node [] (e) {};!
}
\end{lstlisting}
\end{minipage}\hfill
\begin{minipage}{.50\textwidth}
\begin{lstlisting}[escapechar=!, showlines=true, keywords={coroutine, yield}]
coroutine B()
{
    !\tikz[remember picture] \node [] (b) {};! /* ... */
    !\tikz[remember picture] \node [] (c) {};! yield A; 
    !\tikz[remember picture] \node [] (f) {};! /* ... */
}

\end{lstlisting}
\end{minipage}
\begin{tikzpicture}[remember picture, overlay,
    every edge/.append style = { ->, thick, , >=stealth, line width = 1pt }] 
  \draw (a.east) + (0, 0) edge (b.west);
  \draw[orange] (c.west) + (0, 0) edge (d.east);
  \draw[transparent] (e.east) + (0, 0) edge (f.west);
\end{tikzpicture} 
\end{center}
\par
\addtocounter{framenumber}{-1}
\end{frame}


\begin{frame}[fragile]{Korutin (\textit{coroutine})}
A végrehajtás mindig ott folytatódik, ahol a legutóbbi hívás esetén abbamaradt.

A lokális változók értéke megőrződik a hívások között.
\\
\begin{center}
\begin{minipage}{.40\textwidth}
\begin{lstlisting}[escapechar=!, keywords={coroutine, yield}]
coroutine A()
{
    /* ... */
    yield B; !\tikz[remember picture] \node [] (a) {};!
    /* ... */!\tikz[remember picture] \node [] (d) {};!
    yield B; !\tikz[remember picture] \node [] (e) {};!
}
\end{lstlisting}
\end{minipage}\hfill
\begin{minipage}{.50\textwidth}
\begin{lstlisting}[escapechar=!, showlines=true, keywords={coroutine, yield}]
coroutine B()
{
    !\tikz[remember picture] \node [] (b) {};! /* ... */
    !\tikz[remember picture] \node [] (c) {};! yield A; 
    !\tikz[remember picture] \node [] (f) {};! /* ... */
}

\end{lstlisting}
\end{minipage}
\begin{tikzpicture}[remember picture, overlay,
    every edge/.append style = { ->, thick, >=stealth, line width = 1pt }] 
  \draw (a.east) + (0, 0) edge (b.west);
  \draw (c.west) + (0, 0) edge (d.east);
  \draw[orange] (e.east) + (0, 0) edge (f.west);
\end{tikzpicture} 
\end{center}
\par
\addtocounter{framenumber}{-1}
\end{frame}


\begin{frame}[fragile]{Generátor (\textit{generator})}
Aszimmetrikus korutin, amely elemek sorozatát állítja elő.

\hfill \\

Minden egyes hívás alkalmával a sorozat egy elemét képzi.

\begin{center}
\begin{minipage}{.40\textwidth}
\begin{lstlisting}[escapechar=!, showlines=true, keywords={for, in, print}]
for ch in alphabet()
{
    print ch
}



\end{lstlisting}
\end{minipage}\hfill
\begin{minipage}{.50\textwidth}
\begin{lstlisting}[escapechar=!, showlines=true, keywords={generator, yield}]
generator alphabet()
{
    yield 'a';
    yield 'b';
    yield 'c';
    /* ... */
}
\end{lstlisting}
\end{minipage}
\end{center}
\par
\end{frame}

  \section{Generátorok ismert nyelvekben}

\begin{frame}{Támogatást biztosító nyelvek}
    \begin{block}{Első, úttörő próbálkozások}
        \indent\textit{IPL-V}, \textit{Alphard}
    \end{block}
    \begin{block}{A \texttt{yield} kulcsszó bevezetése}
        \textit{CLU}
    \end{block}
    \begin{block}{TIOBE Top 10, 2017. március}
        \begin{itemize}
            \item
            \textit{C\#}
            \item
            \textit{Python}
            \item
            \textit{Visual Basic .NET}
            \item
            \textit{PHP}
            \item
            \textit{JavaScript}
        \end{itemize}
    \end{block}
\end{frame}


\begin{frame}{Felhasználási lehetőségek -- 1}
\begin{block}{Végtelen sorozatok}
    \begin{itemize}
        \item
        Fibonacci sorozat
        \item
        Prímszámok
        \item
        Véletlen értékek forrása
    \end{itemize}
\end{block}
\begin{block}{Véges sorozatok}
    \begin{itemize}
        \item
        Reguláris kifejezésre illesztés eredményei
        \item
        Fájl beolvasása soronként/darabonként
        \item
        Paraméteres görbék pontjai
    \end{itemize}
\end{block}
\end{frame}


\begin{frame}{Felhasználási lehetőségek -- 2}
\begin{itemize}
    \item
    Szimmetrikus korutinok modellezése \\
    \hfill \\
    \item
    \texttt{async/await} szimulálása (\textit{JavaScript}) \\
    \hfill \\
    \item
    Mikroszálak létrehozása, ütemezése
\end{itemize}
\end{frame}

  \section{Korábbi Java implementációk}

\begin{frame}{Általános problémák}
Csak régebbi \textit{Java} verziók támogatása (\textit{Java SE} 6-7)


\hfill \\

\pause
Kényelmetlen interfész


\hfill \\

\pause
Rugalmatlan, kevésbé megbízható megvalósítás
\end{frame}


\begin{frame}[fragile]{jyield}
\begin{center}
\begin{lstlisting}[language=java, xleftmargin=15pt,
        basicstyle=\scriptsize,
        numbers=left,
        numbersep=5pt, escapechar=!]
@Continuable
public Iterable<Integer> power(int number, int exponent) 
{
        int counter = 0;
        int result = 1;
        while (counter++ < exponent) {
                result = result * number;

                System.out.print("[" + result + "]");

                Yield.ret(result);
        }
        return Yield.done();
}
\end{lstlisting}
\end{center}
\par
\end{frame}


\begin{frame}[fragile]{java-generator-functions}
\begin{center}
\begin{lstlisting}[language=java, xleftmargin=15pt,
        basicstyle=\scriptsize,
        numbers=left,
        numbersep=5pt, escapechar=!]
Generator<Integer> simpleGenerator = 
  new Generator<Integer>() {
    public void run() throws InterruptedException {
        yield(1);
        /* ... */
        yield(2);
    }
};
\end{lstlisting}
\end{center}
\par
\end{frame}


\begin{frame}[fragile]{lombok-pg}
\begin{center}
\begin{lstlisting}[language=java, xleftmargin=15pt,
        basicstyle=\scriptsize,
        numbers=left,
        numbersep=5pt, escapechar=!]
public Iterable<Integer>
  power(final int number, final int exponent) 
{
        int counter = 0;
        int result = 1;
        while (counter++ < exponent) {
                result = result * number;

                System.out.print("[" + result + "]");

                yield(result);
        }
}
\end{lstlisting}
\end{center}
\par
\end{frame}

  \section{Az eljárás egy példán keresztül}

\begin{frame}{Az eljárás jellemzői}
A \textit{Pluggable Annotation Processing} API-t használja


\hfill \\

\pause
A \texttt{return} utasítás jelképezi a \texttt{yield} utasítást


\hfill \\

\pause
\textit{Continuation Passing Style}-alapú megvalósítás
\end{frame}


\begin{frame}[fragile]{Prímek generálása}
\begin{center}
\begin{lstlisting}[language=java, xleftmargin=15pt,
        basicstyle=\fontsize{7}{9}\selectfont,
        numbers=left,
        numbersep=5pt, escapechar=!,
        linebackgroundcolor={
            \btLstHL<1>{30} % No highlighting
            \btLstHL<2>{5}
            \btLstHL<3>{20}
        }]
@Generator
private Stream<Integer> generatePrimes() {
    LinkedList<Integer> primes = new LinkedList<>();
    primes.add(2);
    return 2;

    int current = 1;

    loop:
    do {
        current += 2;
        for (int i : primes) {
            if (current % i == 0) {
                continue loop;
            }
        }

        primes.add(current);

        return current;
    } while (true);
}
\end{lstlisting}
\end{center}
\par
\end{frame}


\begin{frame}[fragile]{Prímek generálása -- Vágási pontok}
\begin{center}
\begin{lstlisting}[language=java, xleftmargin=15pt,
        basicstyle=\fontsize{7}{9}\selectfont,
        numbers=left,
        numbersep=5pt, escapechar=!,
        linebackgroundcolor={
            \btLstHL<1>{30} % No highlighting
            \btLstHL<2>{5, 20}
            \btLstHL<3>{10, 21}
            \btLstHL<4>{14}
            \btLstHL<5>{12-16}
        }]
@Generator
private Stream<Integer> generatePrimes() {
    LinkedList<Integer> primes = new LinkedList<>();
    primes.add(2);
    return 2;

    int current = 1;

    loop:
    do {
        current += 2;
        for (int i : primes) {
            if (current % i == 0) {
                continue loop;
            }
        }

        primes.add(current);

        return current;
    } while (true);
}
\end{lstlisting}
\end{center}
\par
\end{frame}


\begin{frame}[t, fragile]{A metódus darabjai}
\noindent
\begin{minipage}[t][0.4\textheight][t]{\textwidth}
    \noindent
    \begin{minipage}{.50\textwidth}
    \noindent
    \begin{center}\textbf{Eredeti kód}\end{center}
    \begin{lstlisting}[language=java, basicstyle=\scriptsize]
    if (current % i == 0) {
        continue loop;
    }
    \end{lstlisting}
    \end{minipage}\hfill
    \begin{minipage}{.50\textwidth}
    \begin{center}\textbf{Continuation}\end{center}
    \begin{center}
        \begin{tabular}{l | c}
            \texttt{if} & 1 \\
            \texttt{else} & 2 \\
            \texttt{continue} & 3 \\
            \texttt{continue loop} & 3 \\
            \texttt{break} & 3 \\
            \texttt{break loop} & 3
        \end{tabular}
    \end{center}
    \end{minipage}
\end{minipage} \\
\begin{minipage}[b][0.6\textheight][t]{\textwidth}
    \begin{center}\textbf{Új darab}\end{center}
    \begin{lstlisting}[language=java, basicstyle=\scriptsize]
    private Bounce<Integer> primes_8(GeneratorState<Integer> contParam) {
        if (current % i == 0) {
            return Bounce.cont(()->primes_4(contParam));
        }
        return Bounce.cont(()->primes_7(contParam));
    }
    \end{lstlisting}
\end{minipage}
\end{frame}


  \begin{frame}{Összefoglalás}
    A generátorok bevezetése \textit{Java}ban indokolt, hiszen rendkívül rugalmasan felhasználhatóak.

    \hfill \\

    A bemutatott eljárás mindössze egy annotáció elhelyezését igényli a programozó részéről.

    \hfill \\

    A háttérben a metódusok kisebb darabokra vágása történik meg, melyeket futásidőben egy \textit{trampoline} vezérel.

    \hfill \\

    A vezérlési szerkezeteket elágazások és megfelelő módon szervezett \textit{continuation}ök modellezik.
  \end{frame}

  \begin{frame}[standout]
    Köszönöm a figyelmet!
  \end{frame}

  % \section{Continuation Passing Style}

\begin{frame}{Mi a \textit{continuation}?}
A hátralevő számításokat reprezentálja, hogy \textit{mi a következő teendő}.

\hfill \\

Absztrakt fogalom, mely többféle módon is realizálható.

\hfill \\

Leggyakrabban egy függvény jelenti a \textit{continuation}t.
\end{frame}

\begin{frame}{\textit{CPS} szabályok}
\begin{enumerate}
\item
A függvények sohasem térhetnek vissza.
\item
Minden függvény paraméterlistája kiegészül egy (vagy több) \textit{continuation}t jelképező paraméterrel.
\end{enumerate}
\end{frame}
\end{document}
